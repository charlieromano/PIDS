\chapter{Introducción específica} % Main chapter title

\label{Chapter2}

%----------------------------------------------------------------------------------------
%	SECTION 1
%----------------------------------------------------------------------------------------
Todos los capítulos deben comenzar con un breve párrafo introductorio que indique cuál es el contenido que se encontrará al leerlo.  La redacción sobre el contenido de la memoria debe hacerse en presente y todo lo referido al proyecto en pasado, siempre de modo impersonal.

\begin{figure}[h]
\centering
\includegraphics[scale=.45]{./Figures/cuadradoAzul.png}
\end{figure}

observarse en la figura \ref{fig:cuadradoAzul}''.

\begin{figure}[ht]
	\centering
	\includegraphics[scale=.45]{./Figures/cuadradoAzul.png}
	\caption{Ilustración del cuadrado azul que se eligió para el diseño del logo.}
	\label{fig:cuadradoAzul}
\end{figure}


\begin{equation}
	\label{eq:metric}
	ds^2 = c^2 dt^2 \left( \frac{d\sigma^2}{1-k\sigma^2} + \sigma^2\left[ d\theta^2 + \sin^2\theta d\phi^2 \right] \right)
\end{equation}

\begin{equation}
	\label{eq:schrodinger}
	\frac{\hbar^2}{2m}\nabla^2\Psi + V(\mathbf{r})\Psi = -i\hbar \frac{\partial\Psi}{\partial t}
\end{equation}

\begin{verbatim}
\begin{equation}
	\label{eq:metric}
	ds^2 = c^2 dt^2 \left( \frac{d\sigma^2}{1-k\sigma^2} + 
	\sigma^2\left[ d\theta^2 + 
	\sin^2\theta d\phi^2 \right] \right)
\end{equation}
\end{verbatim}

Y para la ecuación \ref{eq:schrodinger}:

\begin{verbatim}
\begin{equation}
	\label{eq:schrodinger}
	\frac{\hbar^2}{2m}\nabla^2\Psi + V(\mathbf{r})\Psi = 
	-i\hbar \frac{\partial\Psi}{\partial t}
\end{equation}

\end{verbatim}

\section{Trenes: Redes de comunicación TCN}

La red de comunicaciones del tren (TCN) presenta una arquitectura de buses jerárquicos de dos niveles que se pueden identificar en la figura 1: el bus de datos WTB y el MVB [IEC-61375,1999]. El WTB se encarga de las comunicaciones entre coches a través de nodos con redundancia física, mientras que al bus MVB se conectan los dispositivos de cada coche. Algunos de estos dispositivos son el control de puertas (DOORL/R), el aire acondicionado (HVAC), el sistema de tracción (VVVF), el sistema de control de frenos (BCU), entre otros. El mapa de recorrido y los carteles LED en conjunto con otros dispositivos como los parlantes y las cámaras de video (CCTV) forman un sistema denominado Sistema de información al pasajero (PIDS).


\pagebreak
\section{PIDS: Sistema de información visual para pasajeros de trenes}

\pagebreak
\section{Carteles y controladoras de matrices LED}
