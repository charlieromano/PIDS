%----------------------------------------------------------------------------------------

% Define some commands to keep the formatting separated from the content 
\newcommand{\keyword}[1]{\textbf{#1}}
\newcommand{\tabhead}[1]{\textbf{#1}}
\newcommand{\code}[1]{\texttt{#1}}
\newcommand{\file}[1]{\texttt{\bfseries#1}}
\newcommand{\option}[1]{\texttt{\itshape#1}}
\newcommand{\grados}{$^{\circ}$}

%----------------------------------------------------------------------------------------
%\section{Introducción}
%----------------------------------------------------------------------------------------

% Chapter 1

\chapter{Introducción} % Main chapter title
\label{Chapter1} % For referencing the chapter elsewhere, use \ref{Chapter1} 
\label{Intro}

Los sistemas de información visual para pasajeros están presentes en diversas industrias y aplicaciones. Se encargan de proveer información a pasajeros en  movimiento y tienen un rol fundamental en la industria del transporte.\\

 Las personas se trasladan por tierra o aire usando automóviles, ómnibus, subtes, trenes o aviones, entre otros. Los sistemas de información visual presentan necesidades y soluciones distintas en cada caso. En una autopista se comunican accidentes u obras viales en ejecución usando carteles gigantes con información en tiempo real. Los pasajeros aéreos visualizan la información de arribo, estado o despegue de vuelos en un aeropuerto. A los pasajeros de ómnibus les interesa conocer los tiempos de espera y las líneas en operación al llegar a una estación. Los pasajeros de trenes usan estos sistemas para conocer el destino o la próxima estación cuando están viajando. En algunos casos los carteles están a la intemperie y en  otros dentro de un recinto, pero en general requieren estar sincronizados con los vehículos en movimiento. \\

Los sistemas de información visual para pasajeros (PIDS) tienen principalmente tres componentes: un sistema que genera datos, una red de transmisión y un sistema de pantallas. Dependiendo del dominio de aplicación, las especificaciones de cada sistema tienen distintos requerimientos. Típicamente en los trenes se requiere comunicación en tiempo real, lo que conlleva la adopción de protocolos de datos de tiempo real (RTP). En aplicaciones ferroviarias también es importante la integridad, disponibilidad y confiabilidad de los datos. Pero también existen otros requerimientos de carácter operativo, como el mantenimiento y la facilidad de instalación. Estos últimos aspectos son esenciales en la operación de una formación ferroviaria y tienen impacto directo en el ciclo de vida de un tren.\\

 Los sistemas PIDS instalados en los trenes se interconectan con una red de comunicaciones (TCN). Esta red TCN sigue un estándar y define tanto interfaces eléctricas como protocolos. A la red TCN se conectan dispositivos para el sensado y control de frenos, de puertas, de monitoreo, entre otros, usando una arquitectura jerárquica de buses de datos. La red TCN representa un estándar robusto, maduro, probado y con gran adopción internacional. Sin embargo los sistemas PIDS se presentan sin la necesidad de ser compatibles con los estándares de TCN, al menos hasta la revisión del año 2005. Existen diversas soluciones comerciales de sistemas PIDS, para aplicaciones de entretenimiento por ejemplo, pero se requiere de un trabajo de integración adicional para que funcionen en un tren.\\
 
 En este trabajo se introduce una breve descripción de las redes TCN y su evolución en el tiempo. Para el caso de las formaciones de Trenes Argentinos, que forman el marco de este trabajo, se presenta también el detalle de interconexión TCN-PIDS, el desarrollo de un sistema de control para los carteles led del sistema PIDS y los resultados de las pruebas de campo realizadas en conjunto con la empresa Trenes Argentinos Operaciones (SOFSE). Se ha organizado esta memoria buscando acercar al lector primero los conceptos principales de la aplicación y luego el detalle técnico del diseño del sistema embebido propuesto. \\
  

En el capítulo 1 se introduce al lector a la motivación original del trabajo realizado. Se explica el marco de investigación del que forma parte este proyecto y se presenta el estado del arte en controles de carteles led.\\

En el capítulo 2 se introduce vocabulario técnico específico. Se presenta una descripción del sistema con el foco en la red de comunicaciones TCN, el sistema PIDS, sus interacciones y componentes.\\

En el capítulo 3 se abordan cuestiones de diseño de sistema. Se especifican los requerimientos y casos de uso que se plantean en el espacio problema y también las consideraciones del espacio solución. Se detalla la solución en términos de arquitectura, patrones de software, descripción de componentes e implementación. Se incluye también los planos de los circuitos eléctricos del hardware existente que fueron relevados al realizar este trabajo.\\

En el capítulo 4 se abordan cuestiones relacionadas al entorno real del sistema: visitas técnicas, mediciones realizadas, hardware ad-hoc realizado para las mediciones y un breve análisis de las tramas de datos de la red PIDS existente.\\

En el capítulo 5 se tratan las conclusiones principales del desarrollo, su potencial fabricación en serie y los pasos a seguir para integrar al resto de ramales ferroviarios. En el apéndice de bibliografía se encontrarán las principales referencias técnicas, científicas e institucionales relevantes para este trabajo.\\


\pagebreak
\section{Introducción general}
En este trabajo se desarrolla el sistema de control para carteles de matriz led del sistema de información visual para pasajeros de trenes argentinos. Las formaciones de trenes argentinos cuentan con carteles de matriz led en sus coches, en el frente y en el contrafrente del tren. Todos los carteles se interconectan a una red de comunicación del sistema PIDS por la que que viajan distintos tipos de datos: por ejemplo datos de mapas led, mensajes de audio, información visual para los carteles, o bien video de cámaras de seguridad. En los buses de datos de la red TCN además se comunican datos de sensores de velocidad, de frenado, eventos que indican apertura o cierre de puertas, por citar algunos ejemplos. \\

Los carteles led del sistema PIDS presentan fallas a lo largo de su ciclo de vida. Esto implica tareas de mantenimiento, reparación o reposición. Si bien existen muchos tipos de carteles led disponibles comercialmente, la integración al sistema de comunicaciones del tren es propietaria del fabricante de trenes. Para el caso de trenes argentinos el proveedor está radicado en China, haciendo muy costoso y lento el proceso de reposición o mantenimiento de equipamiento. Por esta razón, el desarrollo local de tecnología para sistemas PIDS es estratégico porque además de desarrollar la industria local extiende la vida útil de los trenes.\\

%Esta necesidad motiva el desarrollo como búsqueda de autonomía tecnológica en áreas de vacancia que pueden ser cubiertas por el sistema científico-tecnológico nacional.

El eje de este trabajo es el desarrollo de un sistema a medida para Trenes Argentinos. La necesidad que prima es generar y brindar al personal de operaciones información necesaria para construir y mantener los sistemas PIDS. Como consecuencia, este trabajo también tiene impacto directo en el pasajero, ya que contribuye a una mejora en la calidad del servicio.\\


\pagebreak
\section{Objetivos y alcance}

El  marco de este trabajo es un Proyecto de Desarrollo Estratégico (PDE) de la Secretaría de Ciencia y Técnica de la Universidad de Buenos Aires (UBACyT). El PDE se titula PDE\_15\_2020 - "Sistema de monitoreo y gestión de la red TCN en formaciones ferroviarias". Las partes que se involucran y forman parte del equipo de trabajo en este proyecto son el Grupo de Investigación en Calidad y Seguridad de las Aplicaciones Ferroviarias (GICSAFE), creado en 2017 en el marco del Consejo Nacional de Investigaciones Científicas y Técnicas (CONICET) de la República Argentina, y la  Operadora Ferroviaria Sociedad del Estado (SOFSE), también conocida como Trenes Argentinos Operaciones. El proyecto está orientado a cubrir necesidades tecnológicas concretas del sistema ferroviario argentino. Este tipo de proyectos son instrumentos de promoción científico-tecnológica que revalorizan e incrementan el aporte de la Universidad al desarrollo socioproductivo.\\

El objetivo principal de este trabajo es diseñar e implementar un sistema de información visual para pasajeros a bordo del tren. El sistema de información visual para pasajeros existente tiene una parte manual y una automática. Cuando el conductor del tren toma cabina para brindar servicio, programa en una pantalla cuáles van a ser las estaciones cabecera. Los nombres de estas estaciones cabecera se visualizan en las marquesinas del frente y contrafrente del tren, como puede verse en la figura \ref{fig:tren}.

\begin{figure}[ht]
	\centering
	\includegraphics[width=1\textwidth]{./Figures/tren.jpg}
	\caption{Foto de una formación operativa de Trenes Argentinos. Se observa el cartel de matriz led frontal que indica el destino Tigre.}
	\label{fig:tren}
\end{figure}


En el interior de los coches también hay carteles led. En estas marquesinas se presentan mensajes a los pasajeros como el nombre de la próxima estación, o la estación arribada (“próxima estación Belgrano”, “estás en estación Belgrano”, por ejemplo). Ésta información se
presenta automáticamente en base a variables de sistema que monitorean el detenimiento del tren, su velocidad y la apertura o cierre de puertas. Esta y otra información de monitoreo y control viaja por una red de comunicación interna del tren que se denomina TCN (Train Communication Network) de acuerdo al estándar que la define \citep{IEC-61375-1}. Este estándar define para la red TCN dos buses jerárquicos donde se conectan los subsistemas electrónicos: el WTB (Wire Train Bus) y el MVB (Multi-Vehicle Bus) \citep{CSN EN 61375-2-1}\citep{IEC 61375-3-1:2012}. El primero es el bus de mayor jerarquía que se conecta entre vagones y que se usa para monitorear cambios topográficos del tren. En el segundo se conectan los sensores y actuadores de cada coche como son los frenos, los controles de puertas, los monitores de velocidad, el sistema de información, etcétera. Los dos buses establecen el uso de interfaces eléctricas usando redes RS485.\\


 El sistema propuesto en este trabajo pretende leer los mensajes de información al pasajero que viajan por la red existente y presentarlos en un display LED. El sistema se compone principalmente de cuatro partes:
 \begin{itemize}
\item display LED
\item placa de control
\item cableado de interconexión
\item firmware del sistema embebido
 \end{itemize}

El diagrama del prototipo se presenta en la figura \ref{fig:diagramaPIDSCIAA}. El display LED matricial representa los carteles de los coches del tren. La placa de control se debe poder conectar a la entrada con al bus de la red RS485 que corresponda y a la salida con un display LED matricial.

\begin{figure}[ht]
	\centering
	\includegraphics[width=1\textwidth]{./Figures/diagramaPIDSCIAA.png}
	\caption{Diagrama de bloques del sistema embebido propuesto basado en la plataforma EDU-CIAA.}
	\label{fig:diagramaPIDSCIAA}
\end{figure}


La placa de control está basada en la plataforma EDU-CIAA \citep{proyecto-ciaa} o en alguna de las plataformas desarrolladas por el CONICET-GICSAFe. La conexión entre el display y la placa así como de la placa con la red TCN deberá ser compatible con el estándar RS-485, definido como capa física de la red TCN. El
firmware a desarrollar se carga a la placa de control usando el puerto USB de una laptop. Este firmware es el responsable de leer los mensajes del sistema de información al pasajero y presentarlos en el display.\\

Las cualidades que debe satisfacer este proyecto son:
\begin{itemize}
\item compatibilidad: debe cumplir con los estándares asociados a la red TCN;
\item practicidad: debe ser de fácil uso para el personal de Trenes Argentinos Operaciones
\end{itemize}

Este proyecto permitirá implementar las funciones de visualización del sistema de información al pasajero sin depender del equipamiento existente. El sistema existente es un equipamiento integrado y propietario, y este proyecto busca desacoplar algunas de sus funciones, las que corresponden a la visualización de información para pasajeros, y presentarlas en un display LED genérico. Por otro lado, permitirá reponer los carteles que en la actualidad quedan fuera de servicio por fallas o pérdida del material original y no pueden ser reparados. De esta manera, el valor principal que aporta este proyecto es contribuir con la sustitución de repuestos faltantes por medio de desarrollo y reducir la dependencia tecnológica de la empresa con los fabricantes. Este proyecto tiene impacto directo en las formaciones ferroviarias existentes que brindan servicio al pasajero todos los días.\\


\pagebreak
\section{Estado del arte}

En esta breve sección se resumen algunas características y aspectos comunes de los sistemas PIDS, tanto para sistemas ferroviarios como para sistemas de transporte integrados. Lejos de ser un estudio sistemático, se pretende orientar al lector en las consideraciones que fueron tenidas en cuenta en este trabajo. Primero se describe el rol que juegan estos sistemas y una noción de su mercado, mencionando aquellos proveedores que se consideraron relevantes por claridad en la información, marca global y diseño conceptual de la solución. Luego se describen algunas soluciones comerciales interesantes y finalmente se presenta usando tablas algunos aspectos técnicos comunes en distintas soluciones. Adicionalmente se mencionan algunos trabajos académicos relevados. Se han elegido como dimensiones de análisis las funcionalidades y servicios que debe ofrecer un sistema PIDS, las características principales de la oferta de carteles electrónicos, y por último las características técnicas de las unidades de control. \\


El cliente de mayor impacto de los servicios que provee un sistema PIDS es la red de transporte (trenes, subtes, metros, ómnibus) de una gran ciudad, debido a su masividad. Lo que se observa en general es que las empresas que proveen sistemas PIDS a las redes metropolitanas de transporte de las grandes ciudades lo hacen bajo formatos distintos. Algunas instalan televisores o pantallas de video, otras carteles led, otras incluyen carteles impresos con algún elemento indicador tipo led, o bien leds en forma de flecha mezclándose con la señalización para indicar nombres de estaciones, pantallas led para desplegar publicidad entre mensajes, etcétera. En algunos países se han realizado esfuerzos durante la última década para que los sistemas PIDS faciliten el acceso a la información del transporte para personas con discapacidades, movilidad reducida y de edades avanzadas. Actualmente los sistemas PIDS se diseñan teniendo en cuenta al pasajero en el centro de todo, buscando ofrecer servicios de información que mejoren la experiencia de viaje.\\

\begin{figure}[h!]
	\centering
	\includegraphics[width=0.49\textwidth]{./Figures/HitachiCartelPIDS.png}
	\includegraphics[width=0.49\textwidth]{./Figures/HitachiDisplayArray.png}
	\caption{Solución de carteles para sistemas PIDS de Hitachi. Consultado en \citep{Hitachi}}
	\label{fig:Hitachi}
\end{figure}

Hitachi ofrece una solución para publicidad de tres pantallas en array que se sincronizan para formar una sola y transmitir video con conectividad WiMAX. Cada uno de estos arreglos los posicionan arriba de las ventanas en ambos lados de los coches, alcanzando el despliegue de hasta dieciocho pantallas sincronizadas por coche, como se puede ver en la figura \ref{fig:Hitachi}. Con esto logran transmitir varios mensajes distintos en simultáneo a los pasajeros sin que tengan que moverse de su asiento.\\


Toshiba ofrece una solución que permite transmitir publicidad e información al pasajero en una misma pantalla LCD en simultáneo. La solución está centrada en la pantalla como dispositivo central, ofreciendo pantallas de 32" y 42", de 1920 x 540 píxeles, full color de hasta 16.7 millones de colores, com amplio ángulo de visión y de gran luminancia\citep{Toshiba}. En la mayoría de los casos las soluciones ofrecidas buscan cubrir tanto la demanda de un sistema PIDS como la oferta de publicidad de cara al pasajero, como es habitual en las estaciones y formaciones ferroviarias.\\



\begin{figure}[h]
	\centering
	\includegraphics[width=0.49\textwidth]{./Figures/ToshibaPIDS.jpg}
	\includegraphics[width=0.49\textwidth]{./Figures/ToshibaDisplayColorOpciones.jpg}
	\caption{Solución de displays LCD para sistemas PIDS de Toshiba.Consultado en \citep{Toshiba}}
	\label{fig:Toshiba}
\end{figure}


El grupo austríaco Trapeze \citep{Trapeze} distingue cuatro tecnologías principales en sistemas PIDS: Led, LCD, canales móviles o apps, y e-ink que es una tecnología de LCD monocromo relativamente nueva. De los factores a tener en cuenta en la selección de carteles se distinguen los ángulos de visión, las condiciones del ambiente donde van instalados, por ejemplo si están a la intemperie o requieren visibilidad con la luz del sol, el tamaño o resolución de los caracteres en pantalla, la selección de colores y su relación con la capacidad estadística de visión de los pasajeros, el housing mecánico, el acceso a controles para personas con movilidad reducida, la alimentación eléctrica y la capacidad de realizar upgrades de sistema de forma remota. \\ 


\begin{figure}[h]
	\centering
	\includegraphics[width=0.32\textwidth]{./Figures/TrapezeStation.jpg}
	\includegraphics[width=0.32\textwidth]{./Figures/TrapezeOnboard.jpg}
	\includegraphics[width=0.32\textwidth]{./Figures/TrapezeTimetable.jpg}
	\caption{Sistema PIDS del proveedor austríaco Trapeze. Consultado en \citep{Trapeze}}
	\label{fig:Trapeze}
\end{figure}

Además se sugiere la importancia de la precisión en la información que ofrece como servicio el sistema PIDS. Si un pasajero recibe el número de anden incorrecto al llegar a la estación muy probablemente perderá el tren, resultando en una mala experiencia de viaje. La interconexión con otros canales de información sobre todo en puntos nodales de transporte también es favorita. Si un pasajero puede anticiparse y ver el tiempo estimado entre una línea de omnibus o de tren antes de llegar a la estación donde hace combinación, entonces puede tomar una mejor elección basada en datos ofrecidos por el sistema PIDS. Estos y otros aspectos de sistema centrados en el usuario se resumen en la tabla \ref{tab:tablaSistemasPIDS}.\\


\begin{table}[]
\centering
\begin{tabular}{ll}
\hline
\textbf{Conectividad}  & \begin{tabular}[c]{@{}l@{}}RS485\\ Ethernet, Fibra Óptica\\ WiFi, WiMax, GPS\\ 2G / 3G / 4G / 5G\end{tabular}                                                                                                                                   \\ \hline
\textbf{Interconexión} & \begin{tabular}[c]{@{}l@{}}App del Tren\\ Información multinodal\\ Ómnibus\\ Tablas de horarios programados\\ Portales de noticias\\ Publicidad\\ Canal de información estatal\end{tabular}                                                     \\ \hline
\textbf{Accesibilidad} & \begin{tabular}[c]{@{}l@{}}Información por audio\\ Ángulos de visión de las pantallas\\ Facilidades para personas en sillas de ruedas\\ Facilidades para personas de edad avanzada\\ Correcto y cuidado sistema de señalización\end{tabular} \\ \hline
\textbf{Información}   & \begin{tabular}[c]{@{}l@{}}Estimación de tiempos precisa\\ Aviso de cortes en tiempo real\\ Buen trackeo de vehículos\\ Conexiones\\ Mensajes de alerta o precauciones\\ Números de emergencia\end{tabular}                                     \\ \hline
\textbf{Matenibilidad} & \begin{tabular}[c]{@{}l@{}}Fácil instalación\\ Costo de reposición\\ Consumo eléctrico\\ Upgrades\end{tabular}                                                                                                                                     \\ \hline
\end{tabular}
\caption{Principales aspectos y servicios asociados que debe ofrecer un sistema PIDS. Elaboración del autor.}
\label{tab:tablaSistemasPIDS}
\end{table}


Según el punto de vista centrado en los carteles se suele tener en cuenta las dimensiones del cartel, la densidad de píxeles por unidad de área, la cantidad de colores o leds por píxel, los niveles de intensidad lumínica, el brillo y contraste, la potencia eléctrica como especificaciones típicas de los carteles de los sistemas PIDS.  El ángulo de visión es una de las variables más consideradas ya que en sistemas PIDS implican el alcance a mayor cantidad de pasajeros de la información en pantalla. En la tabla \ref{tab:tablaDisplays} se presenta un resumen de estas características. Las fuentes consultadas para la elaboración de esta tabla son diversos portales internacionales de distribución de componentes electrónicos.\\



\begin{table}[h!]
\begin{tabular}{|l|l|l|l|l|}
\hline
\textbf{Display}                                                                   & \textbf{LED matricial}                                                                                          & \textbf{LED RGB}                                                                                                                                                                      & \textbf{TFT LCD}                                                                    & \textbf{LCD RGB}                                                                                                                        \\ \hline
\textbf{Colores}                                                                   & \begin{tabular}[c]{@{}l@{}}monocromo \\ bicolor\\ tricolor\\ multicolor \\ (\textless{}10 colores)\end{tabular} & \begin{tabular}[c]{@{}l@{}}desde 256 \\ hasta 16,7M\\ (típicamente)\end{tabular}                                                                                                      & hasta 16.7M                                                                         & \begin{tabular}[c]{@{}l@{}}16.7M \\ (típicamente)\\ \\  1,000M\end{tabular}                                                             \\ \hline
\textbf{\begin{tabular}[c]{@{}l@{}}Ángulo \\ de visión\end{tabular}}               & 110º                                                                                                            & 160º                                                                                                                                                                                  & 120º-140º                                                                           & 178º                                                                                                                                    \\ \hline
\textbf{Intensidad}                                                                & 450 cd/m2                                                                                                       & 1500-2000 cd/m2                                                                                                                                                                       & 350 cd/m2                                                                           & 900 cd/m2                                                                                                                               \\ \hline
\textbf{\begin{tabular}[c]{@{}l@{}}Densidad \\ de píxeles\end{tabular}}            & \begin{tabular}[c]{@{}l@{}}3,9 K\\ 27,7 K\\ 110 K\\ *\end{tabular}                                              & \begin{tabular}[c]{@{}l@{}}P16: 3,9 K\\ P12: 6,9 K\\ P10: 10 K\\ P8:   15,6 K\\ P6:   27,7 K\\ P5:   40 K\\ P4:   62,5 K\\ P3:    111 K\\ P2.5: 160 K\\ P2:    250 K\\ *\end{tabular} & \begin{tabular}[c]{@{}l@{}}29 M/m2 \\ (pixel size\\ 179um  \\ x 192um)\end{tabular} & \begin{tabular}[c]{@{}l@{}}4,26 M/m2\\ (pixel size \\ 484 um \\ x 484 um)\\ \\ 29 M/m2 \\ (pixel size \\ 179um \\ x 192um)\end{tabular} \\ \hline
\textbf{\begin{tabular}[c]{@{}l@{}}Potencia \\ (consumo \\ promedio)\end{tabular}} & \textless 10 W                                                                                                  & \begin{tabular}[c]{@{}l@{}}15 W a \\ 250W$\sim$316W\end{tabular}                                                                                                                      & 20 W                                                                                & 25W                                                                                                                                     \\ \hline
\end{tabular}
	\caption{Principales características de displays para sistemas PIDS.  Elaboración del autor.}
	\label{tab:tablaDisplays}
\end{table}



Todo cartel requiere de un controlador como interfaz que procesa información y la codifica según la lógica que requiera el tipo de cartel. Los controladores de los carteles de matriz led suelen basarse en circuitos digitales, en microcontroladores de 8, 16 o 32 bits o en FPGA. Las tasas de transmisión de datos requieren señales de clock que pueden variar desde algunos KHz hasta cientos de MHz. Los tamaños del buffer de memoria están en función de la cantidad de píxeles que tenga la pantalla. Las interfaces físicas pueden ser periféricos de un microcontrolador, un pin de propósito general o bien puertos USB o HDMI. Los carteles LCD en muchos casos requieren de la transmisión de señales de video. Esto implica mayores costos de implementación que la alternativa LED pero también mayor versatilidad en la programación de contenidos. En la tabla \ref{tab:tablaControladores} se resumen algunas características principales de los requerimientos de los controladores.\\




\begin{table}[h!]
\centering
\begin{tabular}{|l|l|l|l|}
\hline
\textbf{\begin{tabular}[c]{@{}l@{}}Unidad de \\ procesamiento\end{tabular}} & \textbf{MCU 8/16/32 bits}                                                                       & \textbf{FPGA / ADIC / DSP}                                                        & \textbf{CPU / DSP}                                                                                      \\ \hline
\textbf{Clock}                                                              & 1-200 MHz                                                                                       & 10-250 MHz                                                                        & 1-3 GHz                                                                                                 \\ \hline
\textbf{Memory buffer}                                                      & 1 KB                                                                                            & 1-512 MB                                                                          & 1-10 GB                                                                                                 \\ \hline
\textbf{Conectividad}                                                       & \begin{tabular}[c]{@{}l@{}}UART (1-4)\\ USB (1-2)\\ RS485\\ GPIO (1-20)\\ Ethernet\end{tabular} & \begin{tabular}[c]{@{}l@{}}Pmod\\ I/O pins (20-800)\\ Ethernet\\ USB\end{tabular} & \begin{tabular}[c]{@{}l@{}}USB \\ VGA\\ HDMI\\ DVI\\ display Port\\ PCI / PCI-E\\ Ethernet\end{tabular} \\ \hline
\textbf{Programación}                                                       & C / C++/ Assembly                                                                               & VHDL, Verilog, XML                                                                & \begin{tabular}[c]{@{}l@{}}C, C++, Java, \\ Python, XML\end{tabular}                                    \\ \hline
\end{tabular}
\caption{Principales características de controladores de uso general para aplicaciones PIDS. Elaboración del autor.}
\label{tab:tablaControladores}
\end{table}

Una vez instalados, los sistemas PIDS suelen requerir mantenimiento. Muchas veces hay fallas de hardware, como por ejemplo leds que dejan de funcionar, una fuente de alimentación o una memoria que se debe reemplazar. Otras veces se requieren cambios en el software, por ejemplo actualizar el contenido de un mensaje o bien cambiarlo. Los atributos de mantenibilidad, versatilidad, modularidad y confiabilidad en la implementación pueden tener un impacto económico relevante en la operación de un servicio de transporte. Para líneas de trenes que cuentan con muchas formaciones ferroviarias operando en simultáneo, las tareas de actualización pueden ser muy intensivas en términos de horas de trabajo y requerir también capacitaciones técnicas periódicas al personal de mantenimiento. Incluso no todos los dispositivos pueden recibir actualizaciones en producción, esto es, en la locación física donde funcionan. Muchas veces se los necesita desinstalar, llevar a un centro técnico y actualizar fuera de operación, lo que requiere de ventanas de mantenimiento y de tiempos reducidos para realizar tareas que pueden ser susceptibles a errores. Otra forma es enviar un técnico al sitio que pueda conectar algún periférico y actualizar manualmente cada dispositivo.\\

De los trabajos académicos relevados se mencionan aquellos con propuestas del sistema de control que representan distintas tecnologías. En \citep{song2011design} se utiliza el chip AT89C52 para enviar caracteres chinos sobre matrices de 32 x 192 leds de un solo color; en \citep{liu2011design} se implementa una pantalla led RGB de 320 x 240 píxeles que rota 360º permitiendo visualizar imágenes en color por persistencia de visión; en \citep{kurdthongmee2004design} se desarrollan algoritmos sobre FPGA usando búferes de datos para controlar una pantalla LED de 160 x 32 píxeles alcanzando 32,768 colores; en \citep{lin2021active} se presenta el control de un micro display de transistores de película delgada (TFT) usando modulación por ancho de pulso (PWM) alcanzando 256 niveles de color a una frecuencia de refresco de 60 Hz, basado también en FPGA; en \citep{gago2009control} se presenta el control de píxeles virtuales para matrices led multicolor usando flip-flops tipo D. \\


En el diseño e implementación del presente trabajo, los carteles son de matriz led de un solo color y de distintas dimensiones (8x64, 32 x 64, 32 x 128). El control de los carteles tiene como factor común el uso del conjunto de chips digitales 74HC138, 74HC595 y 74HC245. La topología permite interconectar paneles en serie para construir carteles led de distinto tamaño usando la misma lógica de control. \\


