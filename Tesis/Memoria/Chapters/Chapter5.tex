% Chapter Template

\chapter{Conclusiones} % 

En este trabajo se abordó un problema tecnológico de la industria ferroviaria argentina, se desarrolló un sistema embebido, y se realizaron ensayos en las instalaciones de Trenes Argentinos (SOFSE) que aportaron información muy importante para el desarrollo del sistema.\\

La problemática de Trenes Argentinos expuesta en este trabajo, está asociada al mantenimiento del sistema de visualización de información al pasajero (PIDS). Este sistema presenta carteles de matriz led en los coches, y forma parte de un sistema de red más grande que se encarga de las comunicaciones del tren (TCN). La red TCN está ampliamente utilizada en el transporte ferroviario mundial, y los estándares internacionales que la especifican tienen un rol clave en la industria. Si bien los protocolos y componentes de la red TCN están definidos en las normas o especificaciones, el sistema PIDS no está incluído en el estándar de la versión de red instalada en los trenes de SOFSE, siendo una solución propietaria de un fabricante con escasa documentación disponible. Esto motivó la realización de una serie de ensayos en las formaciones ferroviarias operativas, y en las maquetas de los talleres de SOFSE, para relevar aspectos técnicos de su implementación aplicando técnicas de ingeniería inversa.\\

A lo largo del trabajo, se expuso la arquitectura del sistema PIDS existente y su relación con la red TCN. En la solución instalada en los trenes, hay una relación física directa entre algunos de los bloques de la arquitectura y el hardware, que fue relevada durante los ensayos. Por el contrario, algunas de las interconexiones de la arquitectura son únicamente a nivel lógico. Se observaron bloques como el SCU o la PCU, que funcionan como concentradores y procesadores de buses de datos, sin tener un módulo único de hardware asociado, sino una serie de equipos funcionando en conjunto en distintas unidades de rack.  Con la información de las interconexiones, se ensayaron mediciones en distintos puntos de prueba: entre el TCMS y el PIDS por ejemplo, entre PCU y SCU, o entre el SCU y el IDU. \\

El análisis de datos de las mediciones, en particular para los puntos de prueba TCMS-PIDS, presentó consistencia para los encabezados y final de trama con la documentación de TCN disponible. Sin embargo, se observó que el comienzo y final de trama era distinto del anterior para el punto de prueba SCU-IDU, donde el IDU corresponde al hardware de control de los carteles de matriz led. Estas observaciones resultaron compatibles con mediciones realizadas por el personal de SOFSE con anterioridad. Si bien esta información fue suficiente para desarrollar requerimientos, se ha observado también que las tramas contienen información adicional de otros bloques del sistema, aparte de los carteles de matriz led. Aunque no quede claro el contenido completo de las tramas, es decir, todos los bits que forman parte de la carga útil de las tramas recibidas o transmitidas por los módulos IDU, las hipótesis propuestas para el desarrollo del sistema embebido son compatibles con el comportamiento observado.\\


El desarrollo del sistema embebido ha permitido proponer una solución que otorga cierto grado de abstracción y versatilidad para la interpretación de tramas de datos de la red PIDS. El sistema está diseñado para recibir tramas de longitud variable, que representan eventos, a través de un periférico UART usando técnicas de memoria dinámica. Los datos recibidos pueden ser validados y procesados para activar visualizaciones en carteles de matriz led, compatibles con los que hay instalados en los trenes. El mecanismo de visualización de mensajes para los carteles de matriz led, está desacoplado de la recepción y el procesamiento de las tramas. Es decir, por un lado el sistema puede recibir y reaccionar a eventos externos, y por otro puede generar visualizaciones de mensajes precargados, como por ejemplo el nombre de una estación.\\

 En el diseño se abordaron cuestiones de concurrencia, implementando una arquitectura orientadad a eventos. Se desarrollaron módulos que interactúan entre sí de forma dinámica y que responden a eventos asincrónicos, como por ejemplo la recepción de datos vía UART. Así, un objeto recibe un mensaje, otro lo procesa, y otro genera la visualización en el cartel de matriz led. Cada módulo fue implementado utilizando patrones de diseño de software como máquinas de estados, a su vez embebidas en objetos activos, utilizando una interfaz de comunicación estándar a través de colas de mensajes.  Al implementar los patrones de diseño en lenguaje C, se buscó construir un conjunto de plantillas que pueda facilitar y satisfacer atributos de calidad de software como modularidad y escalabilidad. Todos los objetos activos implementados funcionan de forma orquestada por un sistema operativo de tiempo real. Las relaciones entre componentes se han presentado con vistas estructurales y de interacciones, siguiendo lineamientos de modelado de software UML. Estas interacciones, cumplen con la solución para los casos de uso detallados en la etapa de requerimientos. Para el desarrollo se ha utilizado la plataforma de hardware EDU-CIAA, la capa de abstracción de hardware firmwareV3, y el sistema operativo de tiempo real freeRTOS. Estas tecnologías son de código abierto y de hardware abierto, es decir que además de estar mantenidas por la comunidad, son de acceso libre y cualquiera las puede usar y mejorar el diseño original. \\


El sistema embebido desarrollado fue probado en una maqueta. Para desarrollar un producto funcional o comercial a partir de este sistema, hace falta realizar una serie de ensayos adicionales de compatibilidad en las instalaciones de trenes. Por un lado, hace falta resolver cuestiones de compatibilidad eléctrica: la red de trenes tiene una tensión de línea de 110 Volts y el circuito del sistema embebido funciona con 5 Volt. Se ha observado en las placas IDU, distintos módulos conversores de tensión para alimentar el circuito que procesa los datos. El desarrollo de un circuito impreso, incluyendo módulos conversores de tensión, ha quedado fuera del alcance de este trabajo. Por otro lado, el sistema se ha diseñado para dar versatilidad a la hora de interpretar las tramas de datos de la red PIDS, sin conocer su contenido con completitud. Para completar el estudio integral de las tramas de la red PIDS, hace falta ajustar y verificar el procesamiento de las tramas, por ejemplo vía simulación.\\


Como prospectiva, aunque se ha observado que la versión instalada de redes TCN y PIDS en los trenes de SOFSE corresponden a redes RS485, al momento de redactar este documento se conoce de la existencia de una nueva norma de TCN basada en redes Ethernet, que incluye al PIDS (nomenclado como PIS) dentro el nuevo estándar. Se propone por lo tanto, explorar este nuevo estándar para comprender aspectos de compatibilidad de sistema en caso de realizar una migración.\\