%%%%%%%%%%%%%%%%%%%%%%%%%%%%%%%%%%%%%%%%%%%%%%%%%%%%%%%%%%%%%%%%%%%%%%%%%%%%%%%%
%
% Template license:
% CC BY-NC-SA 3.0 (http://creativecommons.org/licenses/by-nc-sa/3.0/)
%
%%%%%%%%%%%%%%%%%%%%%%%%%%%%%%%%%%%%%%%%%%%%%%%%%%%%%%%%%%%%%%%%%%%%%%%%%%%%%%%%

%----------------------------------------------------------------------------------------
%	PACKAGES AND OTHER DOCUMENT CONFIGURATIONS
%----------------------------------------------------------------------------------------

\documentclass[
11pt, % The default document font size, options: 10pt, 11pt, 12pt
%oneside, % Two side (alternating margins) for binding by default, uncomment to switch to one side
%chapterinoneline,% Have the chapter title next to the number in one single line
spanish,
singlespacing, % Single line spacing, alternatives: onehalfspacing or doublespacing
%draft, % Uncomment to enable draft mode (no pictures, no links, overfull hboxes indicated)
%nolistspacing, % If the document is onehalfspacing or doublespacing, uncomment this to set spacing in lists to single
%liststotoc, % Uncomment to add the list of figures/tables/etc to the table of contents
%toctotoc, % Uncomment to add the main table of contents to the table of contents
parskip, % Uncomment to add space between paragraphs
%codirector, % Uncomment to add a codirector to the title page
headsepline, % Uncomment to get a line under the header
]{MastersDoctoralThesis} % The class file specifying the document structure



%----------------------------------------------------------------------------------------
%	INFORMACIÓN DE LA MEMORIA
%----------------------------------------------------------------------------------------

\thesistitle{PIDS: Sistema de información visual para pasajeros de Trenes Argentinos} % El títulos de la memoria, se usa en la carátula y se puede usar el cualquier lugar del documento con el comando \ttitle

% Nombre del posgrado, se usa en la carátula y se puede usar el cualquier lugar del documento con el comando \degreename
\posgrado{Carrera de Especialización en Sistemas Embebidos} 
%\posgrado{Carrera de Especialización en Internet de las Cosas} 
%\posgrado{Carrera de Especialización en Intelegencia Artificial}
%\posgrado{Maestría en Sistemas Embebidos} 
%\posgrado{Maestría en Internet de las cosas}

\author{Ing. Carlos German Carreño Romano} % Tu nombre, se usa en la carátula y se puede usar el cualquier lugar del documento con el comando \authorname

\director{Dr. Ing. Pablo Martín Gomez (UBA)} % El nombre del director, se usa en la carátula y se puede usar el cualquier lugar del documento con el comando \dirname
\codirector{Nombre del codirector (pertenencia)} % El nombre del codirector si lo hubiera, se usa en la carátula y se puede usar el cualquier lugar del documento con el comando \codirname.  Para activar este campo se debe descomentar la opción "codirector" en el comando \documentclass, línea 23.

\juradoUNO{Nombre del jurado 1 (pertenencia)} % Nombre y pertenencia del un jurado se usa en la carátula y se puede usar el cualquier lugar del documento con el comando \jur1name
\juradoDOS{Nombre del jurado 2 (pertenencia)} % Nombre y pertenencia del un jurado se usa en la carátula y se puede usar el cualquier lugar del documento con el comando \jur2name
\juradoTRES{Nombre del jurado 3 (pertenencia)} % Nombre y pertenencia del un jurado se usa en la carátula y se puede usar el cualquier lugar del documento con el comando \jur3name

\ciudad{Ciudad Autónoma de Buenos Aires}
%\ciudad{ciudad de Mendoza}

\fechaINICIO{marzo de 2020}
\fechaFINAL{diciembre de 2023}


\keywords{Sistemas embebidos, FIUBA} % Keywords for your thesis, print it elsewhere with \keywordnames

\begin{document}


\frontmatter % Use roman page numbering style (i, ii, iii, iv...) for the pre-content pages

\pagestyle{plain} % Default to the plain heading style until the thesis style is called for the body content


%----------------------------------------------------------------------------------------
%	RESUMEN - ABSTRACT 
%----------------------------------------------------------------------------------------

\begin{abstract}
\addchaptertocentry{\abstractname} % Add the abstract to the table of contents
%
%The Thesis Abstract is written here (and usually kept to just this page). The page is kept centered vertically so can expand into the blank space above the title too\ldots
\centering

En este trabajo se aborda el desarrollo de un sistema de información visual para pasajeros para la empresa Trenes Argentinos. Las formaciones ferroviarias tienen carteles de matriz led dentro de los coches por donde se comunican mensajes tales como la próxima estación. Estos carteles forman parte de una red de comunicaciones que interconecta distintos sistemas de control dentro del tren. El sistema desarrollado en este trabajo puede recibir datos de la red de comunicaciones y controlar carteles de matriz led compatibles con los existentes. El sistema sigue una arquitectura orientada a eventos, haciendo uso de técnicas de concurrencia y patrones de software tales como objeto activo, máquinas de estado y gestión de memoria dinámica, sobre un sistema operativo de tiempo real. Se han realizado ensayos y mediciones en trenes operativos, que han proporcionado tramas de datos reales que fueron analizadas para usarse como entradas del sistema. En la región del  AMBA, los trenes operan aproximadamente con 500 carteles del mismo modelo de forma simultánea. Estos suelen presentar fallas, y su reemplazo se ve dificultado por problemas de importación. Con este trabajo se pretende contribuir a la sustitución de las placas de control y al mantenimiento de las unidades, con el fin de extender la vida útil de los trenes. 

\end{abstract}

%----------------------------------------------------------------------------------------
%	CONTENIDO DE LA MEMORIA  - AGRADECIMIENTOS
%----------------------------------------------------------------------------------------

\begin{acknowledgements}
%\addchaptertocentry{\acknowledgementname} % Descomentando esta línea se puede agregar los agradecimientos al índice
\vspace{1.5cm}

En primer lugar, quiero agradecer al Dr. Ing. Pablo Gomez por su inagotable paciencia y excelente predisposición a lo largo de todo el tiempo que llevó concluir este trabajo. También quiero agradecer al Dr. Ing. Ariel Lutenberg por su empuje, su espíritu motivador y su actitud de concertación. Me gustaría destacar su liderazgo en la generación de propuestas concretas de vinculación entre la universidad y la industria a través de proyectos de desarrollo de tecnología. Quiero reconocer el valioso aporte de ambos y del cuerpo docente que han volcado a lo largo de los años un enorme y minucioso trabajo en el desarrollo de los contenidos del programa de la Carrera de Especialización en Sistemas Embebidos. Ha sido altamente satisfactorio completar el posgrado.

Asimismo, me gustaría agradecer al personal altamente calificado de la Gerencia de Material Rodante Eléctrico de la companía Trenes Argentinos Operaciones (SOFSE), el Ing. Sergio Dieleke, y a todos los colaboradores que nos han brindado atención, seguridad y tiempo para realizar mediciones en las formaciones ferroviarias. Quiero hacer una mención especial de reconocimiento a Bruno Pilato por su colaboración desde los talleres de Castelar para realizar pruebas en conjunto durante la etapa de confinamiento debido a la pandemia. 

Por último, también agradezco a Magui, que me brindó soporte en 
todo momento; a mi madre y a mi padre por enseñarme a valorar tanto la formación académica como la experiencia técnica con igual importancia; y a mis tutores y colegas de la empresa con quienes comparto el día a día e intercambio ideas sumamente útiles para pensar fuera de la caja.


\end{acknowledgements}

%----------------------------------------------------------------------------------------
%	LISTA DE CONTENIDOS/FIGURAS/TABLAS
%----------------------------------------------------------------------------------------

\tableofcontents % Prints the main table of contents

%\listoffigures % Prints the list of figures

%\listoftables % Prints the list of tables


%----------------------------------------------------------------------------------------
%	CONTENIDO DE LA MEMORIA  - DEDICATORIA
%----------------------------------------------------------------------------------------

%\dedicatory{\textbf{Dedicado a la Facultad de Ingeniería de la Universidad de Buenos Aires y a la empresa Trenes Argentinos Operaciones}}  % escribir acá si se desea una dedicatoria

%----------------------------------------------------------------------------------------
%	CONTENIDO DE LA MEMORIA  - CAPÍTULOS
%----------------------------------------------------------------------------------------

\mainmatter % Begin numeric (1,2,3...) page numbering

\pagestyle{thesis} % Return the page headers back to the "thesis" style

% Incluir los capítulos como archivos separados desde la carpeta Chapters

%----------------------------------------------------------------------------------------

% Define some commands to keep the formatting separated from the content 
\newcommand{\keyword}[1]{\textbf{#1}}
\newcommand{\tabhead}[1]{\textbf{#1}}
\newcommand{\code}[1]{\texttt{#1}}
\newcommand{\file}[1]{\texttt{\bfseries#1}}
\newcommand{\option}[1]{\texttt{\itshape#1}}
\newcommand{\grados}{$^{\circ}$}

%----------------------------------------------------------------------------------------
%\section{Introducción}
%----------------------------------------------------------------------------------------

% Chapter 1

\chapter{Introducción} % Main chapter title
\label{Chapter1} % For referencing the chapter elsewhere, use \ref{Chapter1} 
\label{Intro}

Los sistemas de información visual para pasajeros están presentes en diversas industrias y aplicaciones. Se encargan de proveer información a pasajeros en  movimiento y tienen un rol fundamental en la industria del transporte.\\

 Las personas se trasladan por tierra o aire usando automóviles, ómnibus, subtes, trenes o aviones, entre otros. Los sistemas de información visual presentan necesidades y soluciones distintas en cada caso. En una autopista se comunican accidentes u obras viales en ejecución usando carteles gigantes con información en tiempo real. Los pasajeros aéreos visualizan la información de arribo, estado o despegue de vuelos en un aeropuerto. A los pasajeros de ómnibus les interesa conocer los tiempos de espera y las líneas en operación al llegar a una estación. Los pasajeros de trenes usan estos sistemas para conocer el destino o la próxima estación cuando están viajando. En algunos casos los carteles están a la intemperie y en  otros dentro de un recinto, pero en general requieren estar sincronizados con los vehículos en movimiento. \\

Los sistemas de información visual para pasajeros (PIDS) tienen principalmente tres componentes: un sistema que genera datos, una red de transmisión y un sistema de pantallas. Dependiendo del dominio de aplicación, las especificaciones de cada sistema tienen distintos requerimientos. Típicamente en los trenes se requiere comunicación en tiempo real, lo que conlleva la adopción de protocolos de datos de tiempo real (RTP). En aplicaciones ferroviarias también es importante la integridad, disponibilidad y confiabilidad de los datos. Pero también existen otros requerimientos de carácter operativo, como el mantenimiento y la facilidad de instalación. Estos últimos aspectos son esenciales en la operación de una formación ferroviaria y tienen impacto directo en el ciclo de vida de un tren.\\

 Los sistemas PIDS instalados en los trenes se interconectan con una red de comunicaciones (TCN). Esta red TCN sigue un estándar y define tanto interfaces eléctricas como protocolos. A la red TCN se conectan dispositivos para el sensado y control de frenos, de puertas, de monitoreo, entre otros, usando una arquitectura jerárquica de buses de datos. La red TCN representa un estándar robusto, maduro, probado y con gran adopción internacional. Sin embargo los sistemas PIDS se presentan sin la necesidad de ser compatibles con los estándares de TCN, al menos hasta la revisión del año 2005. Existen diversas soluciones comerciales de sistemas PIDS, para aplicaciones de entretenimiento por ejemplo, pero se requiere de un trabajo de integración adicional para que funcionen en un tren.\\
 
 En este trabajo se introduce una breve descripción de las redes TCN y su evolución en el tiempo. Para el caso de las formaciones de Trenes Argentinos, que forman el marco de este trabajo, se presenta también el detalle de interconexión TCN-PIDS, el desarrollo de un sistema de control para los carteles led del sistema PIDS y los resultados de las pruebas de campo realizadas en conjunto con la empresa Trenes Argentinos Operaciones (SOFSE). Se ha organizado esta memoria buscando acercar al lector primero los conceptos principales de la aplicación y luego el detalle técnico del diseño del sistema embebido propuesto. \\
  

En el capítulo 1 se introduce al lector a la motivación original del trabajo realizado. Se explica el marco de investigación del que forma parte este proyecto y se presenta el estado del arte en controles de carteles led.\\

En el capítulo 2 se introduce vocabulario técnico específico. Se presenta una descripción del sistema con el foco en la red de comunicaciones TCN, el sistema PIDS, sus interacciones y componentes.\\

En el capítulo 3 se abordan cuestiones de diseño de sistema. Se especifican los requerimientos y casos de uso que se plantean en el espacio problema y también las consideraciones del espacio solución. Se detalla la solución en términos de arquitectura, patrones de software, descripción de componentes e implementación. Se incluye también los planos de los circuitos eléctricos del hardware existente que fueron relevados al realizar este trabajo.\\

En el capítulo 4 se abordan cuestiones relacionadas al entorno real del sistema: visitas técnicas, mediciones realizadas, hardware ad-hoc realizado para las mediciones y un breve análisis de las tramas de datos de la red PIDS existente.\\

En el capítulo 5 se tratan las conclusiones principales del desarrollo, su potencial fabricación en serie y los pasos a seguir para integrar al resto de ramales ferroviarios. En el apéndice de bibliografía se encontrarán las principales referencias técnicas, científicas e institucionales relevantes para este trabajo.\\


\pagebreak
\section{Introducción general}
En este trabajo se desarrolla el sistema de control para carteles de matriz led del sistema de información visual para pasajeros de trenes argentinos. Las formaciones de trenes argentinos cuentan con carteles de matriz led en sus coches, en el frente y en el contrafrente del tren. Todos los carteles se interconectan a una red de comunicación del sistema PIDS por la que que viajan distintos tipos de datos: por ejemplo datos de mapas led, mensajes de audio, información visual para los carteles, o bien video de cámaras de seguridad. En los buses de datos de la red TCN además se comunican datos de sensores de velocidad, de frenado, eventos que indican apertura o cierre de puertas, por citar algunos ejemplos. \\

Los carteles led del sistema PIDS presentan fallas a lo largo de su ciclo de vida. Esto implica tareas de mantenimiento, reparación o reposición. Si bien existen muchos tipos de carteles led disponibles comercialmente, la integración al sistema de comunicaciones del tren es propietaria del fabricante de trenes. Para el caso de trenes argentinos el proveedor está radicado en China, haciendo muy costoso y lento el proceso de reposición o mantenimiento de equipamiento. Por esta razón, el desarrollo local de tecnología para sistemas PIDS es estratégico porque además de desarrollar la industria local extiende la vida útil de los trenes.\\

%Esta necesidad motiva el desarrollo como búsqueda de autonomía tecnológica en áreas de vacancia que pueden ser cubiertas por el sistema científico-tecnológico nacional.

El eje de este trabajo es el desarrollo de un sistema a medida para Trenes Argentinos. La necesidad que prima es generar y brindar al personal de operaciones información necesaria para construir y mantener los sistemas PIDS. Como consecuencia, este trabajo también tiene impacto directo en el pasajero, ya que contribuye a una mejora en la calidad del servicio.\\


\pagebreak
\section{Objetivos y alcance}

El  marco de este trabajo es un Proyecto de Desarrollo Estratégico (PDE) de la Secretaría de Ciencia y Técnica de la Universidad de Buenos Aires (UBACyT). El PDE se titula PDE\_15\_2020 - "Sistema de monitoreo y gestión de la red TCN en formaciones ferroviarias". Las partes que se involucran y forman parte del equipo de trabajo en este proyecto son el Grupo de Investigación en Calidad y Seguridad de las Aplicaciones Ferroviarias (GICSAFE), creado en 2017 en el marco del Consejo Nacional de Investigaciones Científicas y Técnicas (CONICET) de la República Argentina, y la  Operadora Ferroviaria Sociedad del Estado (SOFSE), también conocida como Trenes Argentinos Operaciones. El proyecto está orientado a cubrir necesidades tecnológicas concretas del sistema ferroviario argentino. Este tipo de proyectos son instrumentos de promoción científico-tecnológica que revalorizan e incrementan el aporte de la Universidad al desarrollo socioproductivo.\\

El objetivo principal de este trabajo es diseñar e implementar un sistema de información visual para pasajeros a bordo del tren. El sistema de información visual para pasajeros existente tiene una parte manual y una automática. Cuando el conductor del tren toma cabina para brindar servicio, programa en una pantalla cuáles van a ser las estaciones cabecera. Los nombres de estas estaciones cabecera se visualizan en las marquesinas del frente y contrafrente del tren, como puede verse en la figura \ref{fig:tren}.

\begin{figure}[ht]
	\centering
	\includegraphics[width=1\textwidth]{./Figures/tren.jpg}
	\caption{Foto de una formación operativa de Trenes Argentinos. Se observa el cartel de matriz led frontal que indica el destino Tigre.}
	\label{fig:tren}
\end{figure}


En el interior de los coches también hay carteles led. En estas marquesinas se presentan mensajes a los pasajeros como el nombre de la próxima estación, o la estación arribada (“próxima estación Belgrano”, “estás en estación Belgrano”, por ejemplo). Ésta información se
presenta automáticamente en base a variables de sistema que monitorean el detenimiento del tren, su velocidad y la apertura o cierre de puertas. Esta y otra información de monitoreo y control viaja por una red de comunicación interna del tren que se denomina TCN (Train Communication Network) de acuerdo al estándar que la define \citep{IEC-61375-1}. Este estándar define para la red TCN dos buses jerárquicos donde se conectan los subsistemas electrónicos: el WTB (Wire Train Bus) y el MVB (Multi-Vehicle Bus) \citep{CSN EN 61375-2-1}\citep{IEC 61375-3-1:2012}. El primero es el bus de mayor jerarquía que se conecta entre vagones y que se usa para monitorear cambios topográficos del tren. En el segundo se conectan los sensores y actuadores de cada coche como son los frenos, los controles de puertas, los monitores de velocidad, el sistema de información, etcétera. Los dos buses establecen el uso de interfaces eléctricas usando redes RS485.\\


 El sistema propuesto en este trabajo pretende leer los mensajes de información al pasajero que viajan por la red existente y presentarlos en un display LED. El sistema se compone principalmente de cuatro partes:
 \begin{itemize}
\item display LED
\item placa de control
\item cableado de interconexión
\item firmware del sistema embebido
 \end{itemize}

El diagrama del prototipo se presenta en la figura \ref{fig:diagramaPIDSCIAA}. El display LED matricial representa los carteles de los coches del tren. La placa de control se debe poder conectar a la entrada con al bus de la red RS485 que corresponda y a la salida con un display LED matricial.

\begin{figure}[ht]
	\centering
	\includegraphics[width=1\textwidth]{./Figures/diagramaPIDSCIAA.png}
	\caption{Diagrama de bloques del sistema embebido propuesto basado en la plataforma EDU-CIAA.}
	\label{fig:diagramaPIDSCIAA}
\end{figure}


La placa de control está basada en la plataforma EDU-CIAA \citep{proyecto-ciaa} o en alguna de las plataformas desarrolladas por el CONICET-GICSAFe. La conexión entre el display y la placa así como de la placa con la red TCN deberá ser compatible con el estándar RS-485, definido como capa física de la red TCN. El
firmware a desarrollar se carga a la placa de control usando el puerto USB de una laptop. Este firmware es el responsable de leer los mensajes del sistema de información al pasajero y presentarlos en el display.\\

Las cualidades que debe satisfacer este proyecto son:
\begin{itemize}
\item compatibilidad: debe cumplir con los estándares asociados a la red TCN;
\item practicidad: debe ser de fácil uso para el personal de Trenes Argentinos Operaciones
\end{itemize}

Este proyecto permitirá implementar las funciones de visualización del sistema de información al pasajero sin depender del equipamiento existente. El sistema existente es un equipamiento integrado y propietario, y este proyecto busca desacoplar algunas de sus funciones, las que corresponden a la visualización de información para pasajeros, y presentarlas en un display LED genérico. Por otro lado, permitirá reponer los carteles que en la actualidad quedan fuera de servicio por fallas o pérdida del material original y no pueden ser reparados. De esta manera, el valor principal que aporta este proyecto es contribuir con la sustitución de repuestos faltantes por medio de desarrollo y reducir la dependencia tecnológica de la empresa con los fabricantes. Este proyecto tiene impacto directo en las formaciones ferroviarias existentes que brindan servicio al pasajero todos los días.\\


\pagebreak
\section{Estado del arte}

En esta breve sección se resumen algunas características y aspectos comunes de los sistemas PIDS, tanto para sistemas ferroviarios como para sistemas de transporte integrados. Lejos de ser un estudio sistemático, se pretende orientar al lector en las consideraciones que fueron tenidas en cuenta en este trabajo. Primero se describe el rol que juegan estos sistemas y una noción de su mercado, mencionando aquellos proveedores que se consideraron relevantes por claridad en la información, marca global y diseño conceptual de la solución. Luego se describen algunas soluciones comerciales interesantes y finalmente se presenta usando tablas algunos aspectos técnicos comunes en distintas soluciones. Adicionalmente se mencionan algunos trabajos académicos relevados. Se han elegido como dimensiones de análisis las funcionalidades y servicios que debe ofrecer un sistema PIDS, las características principales de la oferta de carteles electrónicos, y por último las características técnicas de las unidades de control. \\


El cliente de mayor impacto de los servicios que provee un sistema PIDS es la red de transporte (trenes, subtes, metros, ómnibus) de una gran ciudad, debido a su masividad. Lo que se observa en general es que las empresas que proveen sistemas PIDS a las redes metropolitanas de transporte de las grandes ciudades lo hacen bajo formatos distintos. Algunas instalan televisores o pantallas de video, otras carteles led, otras incluyen carteles impresos con algún elemento indicador tipo led, o bien leds en forma de flecha mezclándose con la señalización para indicar nombres de estaciones, pantallas led para desplegar publicidad entre mensajes, etcétera. En algunos países se han realizado esfuerzos durante la última década para que los sistemas PIDS faciliten el acceso a la información del transporte para personas con discapacidades, movilidad reducida y de edades avanzadas. Actualmente los sistemas PIDS se diseñan teniendo en cuenta al pasajero en el centro de todo, buscando ofrecer servicios de información que mejoren la experiencia de viaje.\\

\begin{figure}[h!]
	\centering
	\includegraphics[width=0.49\textwidth]{./Figures/HitachiCartelPIDS.png}
	\includegraphics[width=0.49\textwidth]{./Figures/HitachiDisplayArray.png}
	\caption{Solución de carteles para sistemas PIDS de Hitachi. Consultado en \citep{Hitachi}}
	\label{fig:Hitachi}
\end{figure}

Hitachi ofrece una solución para publicidad de tres pantallas en array que se sincronizan para formar una sola y transmitir video con conectividad WiMAX. Cada uno de estos arreglos los posicionan arriba de las ventanas en ambos lados de los coches, alcanzando el despliegue de hasta dieciocho pantallas sincronizadas por coche, como se puede ver en la figura \ref{fig:Hitachi}. Con esto logran transmitir varios mensajes distintos en simultáneo a los pasajeros sin que tengan que moverse de su asiento.\\


Toshiba ofrece una solución que permite transmitir publicidad e información al pasajero en una misma pantalla LCD en simultáneo. La solución está centrada en la pantalla como dispositivo central, ofreciendo pantallas de 32" y 42", de 1920 x 540 píxeles, full color de hasta 16.7 millones de colores, com amplio ángulo de visión y de gran luminancia\citep{Toshiba}. En la mayoría de los casos las soluciones ofrecidas buscan cubrir tanto la demanda de un sistema PIDS como la oferta de publicidad de cara al pasajero, como es habitual en las estaciones y formaciones ferroviarias.\\



\begin{figure}[h]
	\centering
	\includegraphics[width=0.49\textwidth]{./Figures/ToshibaPIDS.jpg}
	\includegraphics[width=0.49\textwidth]{./Figures/ToshibaDisplayColorOpciones.jpg}
	\caption{Solución de displays LCD para sistemas PIDS de Toshiba.Consultado en \citep{Toshiba}}
	\label{fig:Toshiba}
\end{figure}


El grupo austríaco Trapeze \citep{Trapeze} distingue cuatro tecnologías principales en sistemas PIDS: Led, LCD, canales móviles o apps, y e-ink que es una tecnología de LCD monocromo relativamente nueva. De los factores a tener en cuenta en la selección de carteles se distinguen los ángulos de visión, las condiciones del ambiente donde van instalados, por ejemplo si están a la intemperie o requieren visibilidad con la luz del sol, el tamaño o resolución de los caracteres en pantalla, la selección de colores y su relación con la capacidad estadística de visión de los pasajeros, el housing mecánico, el acceso a controles para personas con movilidad reducida, la alimentación eléctrica y la capacidad de realizar upgrades de sistema de forma remota. \\ 


\begin{figure}[h]
	\centering
	\includegraphics[width=0.32\textwidth]{./Figures/TrapezeStation.jpg}
	\includegraphics[width=0.32\textwidth]{./Figures/TrapezeOnboard.jpg}
	\includegraphics[width=0.32\textwidth]{./Figures/TrapezeTimetable.jpg}
	\caption{Sistema PIDS del proveedor austríaco Trapeze. Consultado en \citep{Trapeze}}
	\label{fig:Trapeze}
\end{figure}

Además se sugiere la importancia de la precisión en la información que ofrece como servicio el sistema PIDS. Si un pasajero recibe el número de anden incorrecto al llegar a la estación muy probablemente perderá el tren, resultando en una mala experiencia de viaje. La interconexión con otros canales de información sobre todo en puntos nodales de transporte también es favorita. Si un pasajero puede anticiparse y ver el tiempo estimado entre una línea de omnibus o de tren antes de llegar a la estación donde hace combinación, entonces puede tomar una mejor elección basada en datos ofrecidos por el sistema PIDS. Estos y otros aspectos de sistema centrados en el usuario se resumen en la tabla \ref{tab:tablaSistemasPIDS}.\\


\begin{table}[]
\centering
\begin{tabular}{ll}
\hline
\textbf{Conectividad}  & \begin{tabular}[c]{@{}l@{}}RS485\\ Ethernet, Fibra Óptica\\ WiFi, WiMax, GPS\\ 2G / 3G / 4G / 5G\end{tabular}                                                                                                                                   \\ \hline
\textbf{Interconexión} & \begin{tabular}[c]{@{}l@{}}App del Tren\\ Información multinodal\\ Ómnibus\\ Tablas de horarios programados\\ Portales de noticias\\ Publicidad\\ Canal de información estatal\end{tabular}                                                     \\ \hline
\textbf{Accesibilidad} & \begin{tabular}[c]{@{}l@{}}Información por audio\\ Ángulos de visión de las pantallas\\ Facilidades para personas en sillas de ruedas\\ Facilidades para personas de edad avanzada\\ Correcto y cuidado sistema de señalización\end{tabular} \\ \hline
\textbf{Información}   & \begin{tabular}[c]{@{}l@{}}Estimación de tiempos precisa\\ Aviso de cortes en tiempo real\\ Buen trackeo de vehículos\\ Conexiones\\ Mensajes de alerta o precauciones\\ Números de emergencia\end{tabular}                                     \\ \hline
\textbf{Matenibilidad} & \begin{tabular}[c]{@{}l@{}}Fácil instalación\\ Costo de reposición\\ Consumo eléctrico\\ Upgrades\end{tabular}                                                                                                                                     \\ \hline
\end{tabular}
\caption{Principales aspectos y servicios asociados que debe ofrecer un sistema PIDS. Elaboración del autor.}
\label{tab:tablaSistemasPIDS}
\end{table}


Según el punto de vista centrado en los carteles se suele tener en cuenta las dimensiones del cartel, la densidad de píxeles por unidad de área, la cantidad de colores o leds por píxel, los niveles de intensidad lumínica, el brillo y contraste, la potencia eléctrica como especificaciones típicas de los carteles de los sistemas PIDS.  El ángulo de visión es una de las variables más consideradas ya que en sistemas PIDS implican el alcance a mayor cantidad de pasajeros de la información en pantalla. En la tabla \ref{tab:tablaDisplays} se presenta un resumen de estas características. Las fuentes consultadas para la elaboración de esta tabla son diversos portales internacionales de distribución de componentes electrónicos.\\



\begin{table}[h!]
\begin{tabular}{|l|l|l|l|l|}
\hline
\textbf{Display}                                                                   & \textbf{LED matricial}                                                                                          & \textbf{LED RGB}                                                                                                                                                                      & \textbf{TFT LCD}                                                                    & \textbf{LCD RGB}                                                                                                                        \\ \hline
\textbf{Colores}                                                                   & \begin{tabular}[c]{@{}l@{}}monocromo \\ bicolor\\ tricolor\\ multicolor \\ (\textless{}10 colores)\end{tabular} & \begin{tabular}[c]{@{}l@{}}desde 256 \\ hasta 16,7M\\ (típicamente)\end{tabular}                                                                                                      & hasta 16.7M                                                                         & \begin{tabular}[c]{@{}l@{}}16.7M \\ (típicamente)\\ \\  1,000M\end{tabular}                                                             \\ \hline
\textbf{\begin{tabular}[c]{@{}l@{}}Ángulo \\ de visión\end{tabular}}               & 110º                                                                                                            & 160º                                                                                                                                                                                  & 120º-140º                                                                           & 178º                                                                                                                                    \\ \hline
\textbf{Intensidad}                                                                & 450 cd/m2                                                                                                       & 1500-2000 cd/m2                                                                                                                                                                       & 350 cd/m2                                                                           & 900 cd/m2                                                                                                                               \\ \hline
\textbf{\begin{tabular}[c]{@{}l@{}}Densidad \\ de píxeles\end{tabular}}            & \begin{tabular}[c]{@{}l@{}}3,9 K\\ 27,7 K\\ 110 K\\ *\end{tabular}                                              & \begin{tabular}[c]{@{}l@{}}P16: 3,9 K\\ P12: 6,9 K\\ P10: 10 K\\ P8:   15,6 K\\ P6:   27,7 K\\ P5:   40 K\\ P4:   62,5 K\\ P3:    111 K\\ P2.5: 160 K\\ P2:    250 K\\ *\end{tabular} & \begin{tabular}[c]{@{}l@{}}29 M/m2 \\ (pixel size\\ 179um  \\ x 192um)\end{tabular} & \begin{tabular}[c]{@{}l@{}}4,26 M/m2\\ (pixel size \\ 484 um \\ x 484 um)\\ \\ 29 M/m2 \\ (pixel size \\ 179um \\ x 192um)\end{tabular} \\ \hline
\textbf{\begin{tabular}[c]{@{}l@{}}Potencia \\ (consumo \\ promedio)\end{tabular}} & \textless 10 W                                                                                                  & \begin{tabular}[c]{@{}l@{}}15 W a \\ 250W$\sim$316W\end{tabular}                                                                                                                      & 20 W                                                                                & 25W                                                                                                                                     \\ \hline
\end{tabular}
	\caption{Principales características de displays para sistemas PIDS.  Elaboración del autor.}
	\label{tab:tablaDisplays}
\end{table}



Todo cartel requiere de un controlador como interfaz que procesa información y la codifica según la lógica que requiera el tipo de cartel. Los controladores de los carteles de matriz led suelen basarse en circuitos digitales, en microcontroladores de 8, 16 o 32 bits o en FPGA. Las tasas de transmisión de datos requieren señales de clock que pueden variar desde algunos KHz hasta cientos de MHz. Los tamaños del buffer de memoria están en función de la cantidad de píxeles que tenga la pantalla. Las interfaces físicas pueden ser periféricos de un microcontrolador, un pin de propósito general o bien puertos USB o HDMI. Los carteles LCD en muchos casos requieren de la transmisión de señales de video. Esto implica mayores costos de implementación que la alternativa LED pero también mayor versatilidad en la programación de contenidos. En la tabla \ref{tab:tablaControladores} se resumen algunas características principales de los requerimientos de los controladores.\\




\begin{table}[h!]
\centering
\begin{tabular}{|l|l|l|l|}
\hline
\textbf{\begin{tabular}[c]{@{}l@{}}Unidad de \\ procesamiento\end{tabular}} & \textbf{MCU 8/16/32 bits}                                                                       & \textbf{FPGA / ADIC / DSP}                                                        & \textbf{CPU / DSP}                                                                                      \\ \hline
\textbf{Clock}                                                              & 1-200 MHz                                                                                       & 10-250 MHz                                                                        & 1-3 GHz                                                                                                 \\ \hline
\textbf{Memory buffer}                                                      & 1 KB                                                                                            & 1-512 MB                                                                          & 1-10 GB                                                                                                 \\ \hline
\textbf{Conectividad}                                                       & \begin{tabular}[c]{@{}l@{}}UART (1-4)\\ USB (1-2)\\ RS485\\ GPIO (1-20)\\ Ethernet\end{tabular} & \begin{tabular}[c]{@{}l@{}}Pmod\\ I/O pins (20-800)\\ Ethernet\\ USB\end{tabular} & \begin{tabular}[c]{@{}l@{}}USB \\ VGA\\ HDMI\\ DVI\\ display Port\\ PCI / PCI-E\\ Ethernet\end{tabular} \\ \hline
\textbf{Programación}                                                       & C / C++/ Assembly                                                                               & VHDL, Verilog, XML                                                                & \begin{tabular}[c]{@{}l@{}}C, C++, Java, \\ Python, XML\end{tabular}                                    \\ \hline
\end{tabular}
\caption{Principales características de controladores de uso general para aplicaciones PIDS. Elaboración del autor.}
\label{tab:tablaControladores}
\end{table}

Una vez instalados, los sistemas PIDS suelen requerir mantenimiento. Muchas veces hay fallas de hardware, como por ejemplo leds que dejan de funcionar, una fuente de alimentación o una memoria que se debe reemplazar. Otras veces se requieren cambios en el software, por ejemplo actualizar el contenido de un mensaje o bien cambiarlo. Los atributos de mantenibilidad, versatilidad, modularidad y confiabilidad en la implementación pueden tener un impacto económico relevante en la operación de un servicio de transporte. Para líneas de trenes que cuentan con muchas formaciones ferroviarias operando en simultáneo, las tareas de actualización pueden ser muy intensivas en términos de horas de trabajo y requerir también capacitaciones técnicas periódicas al personal de mantenimiento. Incluso no todos los dispositivos pueden recibir actualizaciones en producción, esto es, en la locación física donde funcionan. Muchas veces se los necesita desinstalar, llevar a un centro técnico y actualizar fuera de operación, lo que requiere de ventanas de mantenimiento y de tiempos reducidos para realizar tareas que pueden ser susceptibles a errores. Otra forma es enviar un técnico al sitio que pueda conectar algún periférico y actualizar manualmente cada dispositivo.\\

De los trabajos académicos relevados se mencionan aquellos con propuestas del sistema de control que representan distintas tecnologías. En \citep{song2011design} se utiliza el chip AT89C52 para enviar caracteres chinos sobre matrices de 32 x 192 leds de un solo color; en \citep{liu2011design} se implementa una pantalla led RGB de 320 x 240 píxeles que rota 360º permitiendo visualizar imágenes en color por persistencia de visión; en \citep{kurdthongmee2004design} se desarrollan algoritmos sobre FPGA usando búferes de datos para controlar una pantalla LED de 160 x 32 píxeles alcanzando 32,768 colores; en \citep{lin2021active} se presenta el control de un micro display de transistores de película delgada (TFT) usando modulación por ancho de pulso (PWM) alcanzando 256 niveles de color a una frecuencia de refresco de 60 Hz, basado también en FPGA; en \citep{gago2009control} se presenta el control de píxeles virtuales para matrices led multicolor usando flip-flops tipo D. \\


En el diseño e implementación del presente trabajo, los carteles son de matriz led de un solo color y de distintas dimensiones (8x64, 32 x 64, 32 x 128). El control de los carteles tiene como factor común el uso del conjunto de chips digitales 74HC138, 74HC595 y 74HC245. La topología permite interconectar paneles en serie para construir carteles led de distinto tamaño usando la misma lógica de control. \\



\chapter{Introducción específica} % Main chapter title

\label{Chapter2}

%----------------------------------------------------------------------------------------
%	SECTION 1
%----------------------------------------------------------------------------------------
Todos los capítulos deben comenzar con un breve párrafo introductorio que indique cuál es el contenido que se encontrará al leerlo.  La redacción sobre el contenido de la memoria debe hacerse en presente y todo lo referido al proyecto en pasado, siempre de modo impersonal.

\begin{figure}[h]
\centering
\includegraphics[scale=.45]{./Figures/cuadradoAzul.png}
\end{figure}

observarse en la figura \ref{fig:cuadradoAzul}''.

\begin{figure}[ht]
	\centering
	\includegraphics[scale=.45]{./Figures/cuadradoAzul.png}
	\caption{Ilustración del cuadrado azul que se eligió para el diseño del logo.}
	\label{fig:cuadradoAzul}
\end{figure}


\begin{equation}
	\label{eq:metric}
	ds^2 = c^2 dt^2 \left( \frac{d\sigma^2}{1-k\sigma^2} + \sigma^2\left[ d\theta^2 + \sin^2\theta d\phi^2 \right] \right)
\end{equation}

\begin{equation}
	\label{eq:schrodinger}
	\frac{\hbar^2}{2m}\nabla^2\Psi + V(\mathbf{r})\Psi = -i\hbar \frac{\partial\Psi}{\partial t}
\end{equation}

\begin{verbatim}
\begin{equation}
	\label{eq:metric}
	ds^2 = c^2 dt^2 \left( \frac{d\sigma^2}{1-k\sigma^2} + 
	\sigma^2\left[ d\theta^2 + 
	\sin^2\theta d\phi^2 \right] \right)
\end{equation}
\end{verbatim}

Y para la ecuación \ref{eq:schrodinger}:

\begin{verbatim}
\begin{equation}
	\label{eq:schrodinger}
	\frac{\hbar^2}{2m}\nabla^2\Psi + V(\mathbf{r})\Psi = 
	-i\hbar \frac{\partial\Psi}{\partial t}
\end{equation}

\end{verbatim}

\section{Trenes: Redes de comunicación TCN}

La red de comunicaciones del tren (TCN) presenta una arquitectura de buses jerárquicos de dos niveles que se pueden identificar en la figura 1: el bus de datos WTB y el MVB [IEC-61375,1999]. El WTB se encarga de las comunicaciones entre coches a través de nodos con redundancia física, mientras que al bus MVB se conectan los dispositivos de cada coche. Algunos de estos dispositivos son el control de puertas (DOORL/R), el aire acondicionado (HVAC), el sistema de tracción (VVVF), el sistema de control de frenos (BCU), entre otros. El mapa de recorrido y los carteles LED en conjunto con otros dispositivos como los parlantes y las cámaras de video (CCTV) forman un sistema denominado Sistema de información al pasajero (PIDS).


\pagebreak
\section{PIDS: Sistema de información visual para pasajeros de trenes}

\pagebreak
\section{Carteles y controladoras de matrices LED}
 
\chapter{Diseño e implementación} % Main chapter title
\label{Chapter3} % Change X to a consecutive number; for referencing this chapter elsewhere, use \ref{ChapterX}

En este capítulo, se abordan cuestiones de diseño, se presentan los requerimientos del sistema y se describe la solución propuesta. Se detalla la solución en términos de arquitectura, patrones de software, descripción de componentes e implementación. En el desarrollo, se utiliza la plataforma de hardware EDU-CIAA \citep{CIAA}, la API Firmware\_ v3 \citep{firmwarev3}, y freeRTOS \citep{freeRTOS} como sistema operativo de tiempo real.\\

% Siguiendo lineamientos de ingeniería de software, se presentan los requerimientos, funcionalidades y casos de uso como ejes principales del espacio problema. Luego se presenta la arquitectura, la organización del código fuente, los patrones implementados, diagramas de secuencia y diagramas de interacción entre componentes como detalle del espacio solución.\\


\definecolor{mygreen}{rgb}{0,0.6,0}
\definecolor{mygray}{rgb}{0.95,0.95,0.95}
\definecolor{mygray50}{rgb}{0.5,0.5,0.5}
\definecolor{mymauve}{rgb}{0.58,0,0.82}

%%%%%%%%%%%%%%%%%%%%%%%%%%%%%%%%%%%%%%%%%%%%%%%%%%%%%%%%%%%%%%%%%%%%%%%%%%%%%
% parámetros para configurar el formato del código en los entornos lstlisting
%%%%%%%%%%%%%%%%%%%%%%%%%%%%%%%%%%%%%%%%%%%%%%%%%%%%%%%%%%%%%%%%%%%%%%%%%%%%%
\lstset{ %
  backgroundcolor=\color{mygray},   % choose the background color; you must add \usepackage{color} or \usepackage{xcolor}
  basicstyle=\footnotesize,        % the size of the fonts that are used for the code
  breakatwhitespace=false,         % sets if automatic breaks should only happen at whitespace
  breaklines=true,                 % sets automatic line breaking
  captionpos=b,                    % sets the caption-position to bottom
  commentstyle=\color{mygreen},    % comment style
  deletekeywords={...},            % if you want to delete keywords from the given language
  %escapeinside={\%*}{*)},          % if you want to add LaTeX within your code
  %extendedchars,=true,              % lets you use non-ASCII characters; for 8-bits encodings only, does not work with UTF-8
  %frame=single,	                % adds a frame around the code
  keepspaces=true,                 % keeps spaces in text, useful for keeping indentation of code (possibly needs columns=flexible)
  keywordstyle=\color{blue},       % keyword style
  language=[ANSI]C,                % the language of the code
  %otherkeywords={*,...},           % if you want to add more keywords to the set
  numbers=left,                    % where to put the line-numbers; possible values are (none, left, right)
  numbersep=5pt,                   % how far the line-numbers are from the code
  numberstyle=\tiny\color{mygray50}, % the style that is used for the line-numbers
  rulecolor=\color{black},         % if not set, the frame-color may be changed on line-breaks within not-black text (e.g. comments (green here))
  showspaces=false,                % show spaces everywhere adding particular underscores; it overrides 'showstringspaces'
  showstringspaces=false,          % underline spaces within strings only
  showtabs=false,                  % show tabs within strings adding particular underscores
  stepnumber=1,                    % the step between two line-numbers. If it's 1, each line will be numbered
  stringstyle=\color{mymauve},     % string literal style
  tabsize=2,	                   % sets default tabsize to 2 spaces
  title=\lstname,                  % show the filename of files included with \lstinputlisting; also try caption instead of title
  morecomment=[s]{/*}{*/}
}
\lstdefinestyle{nonumbers}
{numbers=none}
  


\section{Requerimientos}

 El objetivo principal de este trabajo es diseñar e implementar un sistema de información visual para pasajeros a bordo del tren. Está dirigido a:
\begin{enumerate}
\item Todos los  miembros del grupo de trabajo GICSAFE y SOFSE que participan de proyectos orientados a cubrir necesidades tecnológicas del sistema ferroviario argentino.
\item Alumnos y personal académico con intenciones de participar en proyectos de desarrollo aplicados a la industria.
\item Desarrolladores de software y equipamiento para trenes.
\end{enumerate}

A nivel general, los requerimientos del proyecto son los siguientes:

\begin{itemize}

\item El sistema debe leer datos de información al pasajero de la red interna de los trenes y presentarlos en un display led. El sistema no se encargará de presentar los mensajes en formato de audio.

\item El sistema permitirá implementar las funciones de visualización del sistema de información al pasajero existente. La solución existente es un sistema propietario que integra también un sistema de audio, un CCTV usando cámaras de seguridad, entre otras funcionalidades. 

\item El sistema que se especifica busca desacoplar funciones del equipamiento propietario para permitir realizar tareas de mantenimiento. Como ejemplo, la reposición de carteles led que en la actualidad quedan fuera de servicio por fallas o pérdida del material original y que no pueden ser reparados. 

\end{itemize}

Estos requerimientos generales se traducen en requerimientos específicos y se dividen en tres grupos que se detallan a continuación: requerimientos funcionales, de integración y de documentación. \\

\pagebreak

\textbf{Requerimientos funcionales}

\begin{itemize}
\item El sistema debe controlar arreglos de matrices led de 8x8 (64 leds individuales).
\item El sistema debe presentar en el display información dinámicamente.
\item El sistema debe poder almacenar una cantidad de información para visualización.
\item El sistema debe permitir elegir entre distintos mensajes de visualización.
\item El sistema debe permitir cargar los mensajes a visualizar a través de una computadora.
\item El sistema debe poder reaccionar a un mensaje que es enviado para visualizar.
\end{itemize}

\textbf{Requerimientos de integración con la red TCN
}\begin{itemize}
\item Las placas de control deben ser compatibles con el sistema PIDS existente.
\item Las placas de control deben poder alimentarse con 110 VDC.
\item El bus de datos de entrada debe ser una interfaz RS-485.
\item El sistema debe interpretar las tramas del PIDS que corresponden a los módulos LDU.
\item El sistema debe manejar tramas en ciclos típicos de 16-20 ms.
\end{itemize}

\textbf{Requerimientos de documentación
}\begin{itemize}
\item Se debe generar un documento de casos de prueba.
\item Se debe generar una guía de usuario.
\item Se debe generar una presentación del sistema.
\item Se debe generar un informe final de proyecto.
\end{itemize}

La tabla \ref{tab:Reqs} sintetiza los requerimientos y les asigna un código de referencia para la evaluación de su cumplimiento.
	
\begin{table}[htb]
\caption{Tabla de requerimientos funcionales, de integración y de documentación del trabajo.}
\label{tab:Reqs}
\begin{center}
\begin{tabular}{ll}
\toprule
\textbf{Código} & \textbf{Descripción}                                 \\ \midrule
PIDS-REQ-FN-01  & Control de módulos de matriz led 8x8.                 \\ \hline
PIDS-REQ-FN-02  & Control de paneles de 2x6 modulos.                    \\ \hline
PIDS-REQ-FN-03  & Control de displays basados en arreglos de 3 paneles. \\ \hline
PIDS-REQ-FN-04  & Visualización de mensajes en idioma castellano.       \\ \hline
PIDS-REQ-FN-05  & Visualización de mensajes dinámicos.                  \\ \hline
PIDS-REQ-FN-06  & Almacenamiento de información de trayecto.            \\ \hline
PIDS-REQ-FN-07  & Selección de contenidos disponibles.                  \\ \hline
PIDS-REQ-FN-08  & Upstream de mensajes desde una computadora personal.  \\ \hline
PIDS-REQ-INT-01 & Compatibilidad de sistema con sistema existente.      \\ \hline
PIDS-REQ-INT-02 & Compatibilidad eléctrica.                             \\ \hline
PIDS-REQ-INT-03 & Compatibilidad de interfaces RS485.                   \\ \hline
PIDS-REQ-INT-04 & Compatibilidad con tramas de datos del módulo LDU.    \\ \hline
PIDS-REQ-INT-05 & Procesamiento de tramas menor a 16 ms.                \\ \hline
PIDS-REQ-DOC-01 & Documentación de casos de prueba.                     \\ \hline
PIDS-REQ-DOC-02 & Guía de usuario.                                      \\ \hline
PIDS-REQ-DOC-03 & Presentación del sistema.                             \\ \hline
PIDS-REQ-DOC-04 & Informe final de proyecto.                            \\ \bottomrule
\end{tabular}
\end{center}
\end{table}


Por último se explicita que para el desarrollo del presente proyecto se asume que:

\begin{itemize}
\item No habrá dependencias directas con otros proyectos enmarcados en el mismo PDE \citep{PDE-TCN}.
\item No habrá dificultad ni excesivas demoras en la compra de los componentes electrónicos o
de software necesarios.
\item Se contará con recursos y materiales necesarios para validar las pruebas realizadas.
Trenes Argentinos dará acceso a una formación ferroviaria con red TCN para realizar
pruebas de campo.
\item El sistema de información al pasajero se va a instalar en el sistema PIDS existente de las formaciones ferroviarias en operación.
\item El sistema de información al pasajero no se va a instalar en redes TCN de tiempo real
basadas en Ethernet (ETB/ECN).
\end{itemize}


\section{Casos de uso}
Los casos de uso planteados se presentan como respuesta a historias de usuario. Las historias de usuario propuestas en este trabajo son:
\begin{itemize}
\item Como usuario del tren, quiero ver el nombre de la estación a la que estoy arribando.
\item Como conductor del tren, quiero elegir el destino y recorrido asociado que se visualizará en los coches.
\item Como sistema vinculado, quiero transmitir mensajes de asistencia, emergencia e información al pasajero.
\item Como componente de sistema, quiero recibir e interpretar tramas de la red de datos del sistema PIDS
\end{itemize}

Estas historias de usuario presentan cuatro tipos distintos de usuarios: pasajeros, conductores, sistemas de información al pasajero y componentes internos del sistema. Con esta variedad de usuarios de sistema, se busca definir funcionalidades y casos de uso. Los principales casos de uso del sistema se presentan en la tabla \ref{tab:UseCases}. \\

\begin{table}[htb]
\caption{Tabla de casos de uso.}
\label{tab:UseCases}
\begin{center}
\begin{tabular}{ll}
\toprule
\textbf{Código} & \textbf{Descripción}     \\ 
\midrule
PIDS-UC-01  & Visualizar estación.         \\ \hline
PIDS-UC-02  & Elegir destino.             \\ \hline
PIDS-UC-03  & Información de asistencia. \\ \hline
PIDS-UC-04  & Receptor de tramas.       \\ 
\bottomrule
\end{tabular}
\end{center}
\end{table}

En el caso de uso UC-1, se involucra al tren como sistema disparador cuando arriba a una estación, y se presenta información visual al pasajero usando los carteles led de salón. 

En el caso de uso UC-2, se resuelve una acción del conductor al presionar un botón, y también se presenta información visual al pasajero, en este caso las estaciones cabecera del recorrido que se visualizan en los carteles led de frente y contrafrente del tren. 

El tercer caso de uso, UC-3, presenta información de asistencia cargada previamente, que se dispara por acción de un temporizador mientras el tren está en circulación. 

Por último, el caso de uso UC-4 involucra al módulo SCU (ver figura  \ref{fig:diagramaPIDS} de la red PIDS) y a un sistema externo, como podría ser la computadora de un operador u otro componente de la red. Esto se realiza para codificar las tramas de datos recibidas desde el SCU.\\

\section{Arquitectura del sistema}

El sistema PIDS de Trenes Argentinos es parte de una solución integral de la red TCN del fabricante de los trenes, la empresa China State Railway Group Company, Ltd. En este capítulo, se describen los aspectos más relevantes del sistema PIDS de esa arquitectura y su relación con los componentes del sistema propuesto en este trabajo. El sistema diseñado en este trabajo sigue una arquitectura orientada a eventos. Se desarrollaron e implementaron distintos patrones de software buscando satisfacer propiedades de modularidad, portabilidad y escalabilidad. Algunos de los objetos implementados se describen a nivel de detalle para resaltar criterios y decisiones de diseño relevantes. La relación entre la arquitectura existente del PIDS de trenes y la arquitectura del sistema embebido basado en la plataforma CIAA, busca por un lado satisfacer la compatibilidad eléctrica del hardware, y por otro, las historias de usuario planteadas en los casos de uso, en la etapa de definición de requerimientos. \\

En las secciones siguientes, se describen los componentes del sistema y sus interacciones de manera más detallada. \

\subsection{Contexto de la solución}

El sistema PIDS forma parte de una solución integral: la red de comunicaciones del tren o red TCN. Este sistema proporciona información a los pasajeros y puede ser operado tanto por el conductor del tren como por los operadores. Se representa a nivel sistema con en el diagrama de la figura \ref{fig:diagTrenTcnPids}.\\

\begin{figure}[ht]
	\centering
	\includegraphics[width=0.66\textwidth]{./Figures/diagTrenTcnPids.png}
	\caption{Diagrama del sistema Tren-TCN-PIDS.}
	\label{fig:diagTrenTcnPids}
\end{figure}

La red TCN define una comunicación estándar usando dos buses jerárquicos llamados WTB (\textit{Wire Train Bus}) y MVB (\textit{Multifunction Vehicle Bus}). El sistema PIDS se interconecta al bus de datos MVB, como se indica en el diagrama de la figura \ref{fig:diagTcnPidsBuusesWtbMvb}.\\


\begin{figure}[ht]
	\centering
	\includegraphics[width=0.66\textwidth]{./Figures/diagTcnPidsBusesWtbMvb.png}
	\caption{Diagrama de interconexión TCN-PIDS}
	\label{fig:diagTcnPidsBuusesWtbMvb}
\end{figure}

El sistema PIDS tiene un bus de comunicación propio a través de una red RS485. Uno de los componentes de esta red es el módulo SCU (descrito en la topolgía del PIDS en el capítulo 2), al cual se conectan distintos dispositivos, como los display led, los mapas de recorrido led, las cámaras y los parlantes, tal como se indica en la figura 	\ref{fig:diagPidsScuDevices}.\\


\begin{figure}[ht]
	\centering
	\includegraphics[width=0.66\textwidth]{./Figures/diagPidsScuDevices.png}
	\caption{Diagrama del módulo SCU en la red PIDS.}
	\label{fig:diagPidsScuDevices}
\end{figure}

Al módulo SCU se conectan las unidades IDU, que corresponden a los display led de salón. Cada unidad IDU contiene un controlador (\textit{driver}) y el arreglo de módulos de matriz led que conforman el display. En la figura \ref{fig:diagScuDriverDisplay} se representan estos bloques funcionales.\\


\begin{figure}[ht]
	\centering
	\includegraphics[width=0.66\textwidth]{./Figures/diagScuDriverDisplay.png}
	\caption{Diagrama de bloques del sistema SCU, placa de control y carteles led de salón.}
	\label{fig:diagScuDriverDisplay}
\end{figure}

El alcance del sistema desarrollado en este trabajo cubre la funcionalidad de este conjunto de bloques \textit{Driver + Display}, que en la nomenclatura del sistema PIDS existente corresponde a los módulos IDU. Existen dos unidades de estos módulos por cada salón o coche. Muchas de las formaciones de SOFSE disponen de siete coches, por lo que se tiene un mínimo de catorce unidades IDU (dos por salón), más dos displays externos adicionales para el frente y contrafrente del tren que indican las estaciones cabecera del recorrido. Esto resulta en un total de al menos dieciséis unidades de control de carteles de matriz led por cada tren. Teniendo en cuenta las formaciones operativas de las líneas Mitre, Sarmiento y Roca del Área Metropolitana de Buenos Aires (AMBA), se puede estimar alrededor de treinta trenes operando en simultáneo, lo que resulta en aproximadamente 500 unidades de carteles operando en vivo. El impacto que puede tener el aporte de este trabajo estará directamente relacionado con la operación de Trenes Argentinos, y puede contribuir a la extensión de la vida útil de los trenes.\\

\subsection{Diseño del sistema}

El diseño propuesto tiene como objetivo abordar las funcionalidades del bloque de control del display led. Mientras que el display led es una unidad que se puede adquirir comercialmente, el controlador para la red PIDS es una solución propietaria del fabricante que se pretende reemplazar con este desarrollo. En la figura \ref{fig:diagVistaReDisenhoEduCIAA} se presenta un diagrama de bloques del sistema de control propuesto. Este controlador utiliza la comunicación serie a través de interfaces UART-RS485 y UART-USB. La UART (\textit{Universal Asynchronous Receiver Transmitter}) es un periférico del microcontrolador de la plataforma CIAA. Dado que la alimentación de la CIAA difiere de la existente en la red RS485 de Trenes Argentinos, se requiere un bloque de conversión de tensión DC-DC para garantizar compatibilidad eléctrica. La comunicación con el display se realiza mediante un adaptador eléctrico, que consiste principalmente en un puerto de entrada-salida.\\


\begin{figure}[ht]
	\centering
	\includegraphics[width=1\textwidth]{./Figures/diagVistaReDisenhoEduCIAA.png}
	\caption{Diagrama de bloques del controlador propuesto.}
	\label{fig:diagVistaReDisenhoEduCIAA}
\end{figure}

A nivel lógico, el sistema que propuesto se compone de cuatro objetos activos que interactúan entre sí, tal como se indica en la figura \ref{fig:diagVistaDisenho}. El prefijo '\underline{ao:}' señala que el bloque es un objeto activo. 
\begin{itemize}
\item El objeto activo UART es el encargado de recibir e interpretar las tramas de datos que viajan por la red RS485 del bus de datos del SCU. 
\item El objeto activo Button es el control manual del operador para accionar el sistema. 
\item El objeto activo PIDS es una máquina de estados que representa el estado del tren, y contiene los mensajes y nombres de las estaciones. 
\item El objeto activo DisplayLed es el encargado de codificar los mensajes en el formato correspondiente a la matriz led del display.\\
\end{itemize}

\begin{figure}[ht]
	\centering
	\includegraphics[width=0.8\textwidth]{./Figures/diagVistaDisenho.png}
	\caption{Vista estructural del sistema propuesto.}
	\label{fig:diagVistaDisenho}
\end{figure}



Esta vista estructural del sistema describe la interacción entre componentes e indica su dependencia funcional. El diseño modular permite realizar cambios en los componentes de forma independiente. Por ejemplo, se podrían reemplazar los mensajes al pasajero en el objeto PIDS sin afectar el resto del sistema. En caso de ser necesario cambiar el hardware de control, se podría reemplazar la lógica de control del display led modificando únicamente el objeto Displayled.\\


\section{Implementación del sistema embebido}

En este trabajo, se lleva a cabo la implementación de un sistema embebido utilizando el lenguaje de programación C y el sistema operativo de tiempo real freeRTOS. La plataforma de hardware utilizada es la CIAA-EDU-NXP, que dispone de un microcontrolador LPC4337 de la companía NXP \citep{NXPLPC4337}, con una arquitectura de 32 bits ARM Cortex-M4/M0. El firmware se desarrolló utilizando la \textit{SAPI} y el \textit{firmwarev3} \citep{firmwarev3} como capa de abstracción de hardware. Esta API proporciona una interfaz para utilizar las funciones de la biblioteca CMSIS del fabricante del microcontrlador y es un componente fundamental del proyecto CIAA.\\
 
Durante el diseño del sistema, se crearon plantillas para implementar patrones de software, como máquinas de estado y objetos activos. Los aspectos de calidad de software, como documentación, pruebas unitarias, mantenimiento y escalabilidad,  se tuvieron en cuenta desde el diseño, y su implementación se facilitó mediante el uso de estas plantillas. Los lineamientos y fragmentos de código relevantes de estas plantillas son descriptos en la sección de patrones de software. \\

La implementación de los objetos activos se detalla en la sección de componentes de sistema. Los atributos clave de la solución y las interacciones entre los componentes forman parte de la documentación presentada. Paar una comprensión más profunda, la implementación del controlador del diplay led incluye una sección con detalles adicionales. El firmware busca ser portable a aquellas versiones de hardware de display de matriz led que utilicen los conjuntos de chips 74HC245, 74HC595 y 74HC138, o sistemas digitales equivalentes. \\



\subsection{Organización del código fuente}
La organización del código fuente que conforma el sistema embebido de este trabajo se detalla en el siguiente árbol de archivos:

\begin{lstlisting}[caption=Árbol de archivos del código fuente del sistema., language=Bash, 
	backgroundcolor=\color{mygray},
	]
AppRTOS
|-- config.mk
|-- inc
|   |-- common.h
|   |-- displayled.h
|   |-- FreeRTOSConfig.h
|   |-- ISR_GPIO.h
|   |-- ISR_UART.h
|   |-- main.h
|   |-- modulePanelDisplay.h
|   |-- portmap.h
|   |-- statemachine_AB.h
|   |-- statemachine_button.h
|   |-- statemachine_displayled.h
|   |-- statemachine_PIDS.h
|   |-- statemachine_UART.h
|   `-- userTasks.h
|-- LICENSE.txt
`-- src
    |-- displayled.c
    |-- ISR_GPIO.c
    |-- ISR_UART.c
    |-- main.c
    |-- statemachine_AB.c
    |-- statemachine_button.c
    |-- statemachine_displayled.c
    |-- statemachine_PIDS.c
    |-- statemachine_UART.c
    `-- userTasks.c
\end{lstlisting}

Se pueden observar dos directorios principales: \textit{inc} y \textit{src}. En \textit{inc} se incluyen los archivos de encabezados y en \textit{src} los archivos de código ejecutable. Existe un archivo principal o \textit{main} que es el que instancia las secuencias de incialización, recursos y tareas del sistema operativo. Las tareas, que incluyen a los objetos activos y procesos del sistema, se implementan en los archivos \textit{userTasks}. Luego, para cada implementación de objeto activo existe una máquina de estados asociada en un archivo de encabezados (.h),  y un archivo ejecutable (.c) con el prefijo \textit{statemachine}. Finalmente, los archivos con el prefijo \textit{ISR} corresponden a las rutinas de interrupción. Como cada máquina de estado utiliza recursos del sistema operativo, se ha creado un archivo \textit{common.h} que declara aquellos recursos como colas, \textit{handlers} y semáforos que se utilizan entre tareas para comunicación interprocesos.\\

Esta organización permite encapsulamiento y la modularidad de los componentes del sistema, lo que facilita la escalabilidad y el mantenimiento del sistema.\\

\subsection{Uso de recursos en RTOS}

En este desarrollo, se utiliza freeRTOS, una versión en C de un \textit{kernel} de sistema operativo de tiempo real. La implementación se basa en la inclusión de los archivos de cabecera \textit{FreeRTOS.h} y \textit{FreeRTOSConfig.h}. Esta implementación ha sido validada en arquitecturas ARM Cortex-M4, específicamente en la plataforma EDU-CIAA. 

En todos los casos que fue posible, se utilizaron semáforos, colas y \textit{mutex} para organizar y proteger el uso compartido de recursos. El caso de uso típico de \textit{mutex} es la interacción de distintas tareas con la misma interfaz UART-USB para imprimir mensajes por pantalla. En el código fuente de cada objeto implementado, se utiliza la siguiente plantilla para proteger el recurso, evitando accesos múltiples y posibilidad de \textit{deadlock}:\\

\begin{lstlisting}[caption=Ejemplo de protección de recursos compartidos.,
	language=C, 
	backgroundcolor=\color{mygray},
	]
if (pdTRUE == xSemaphoreTake( xMutexUART, portMAX_DELAY)){
   vPrintString("Task AB is running.\r\n");
   xSemaphoreGive(xMutexUART);
}
\end{lstlisting}

Las colas, conocidas como \textit{Queue}, se utilizan principalmente para comunicar eventos entre objetos activos. En todos los casos se referencian con \textit{handlers} usando el prefijo \textit{queueHandle} como nomenclatura. En el siguiente fragmento de código se muestra un ejemplo, exibiendo los mecanismos de control de errores utilizados. \\

\begin{lstlisting}[caption=Ejemplo de creación de cola.,
	language=C, 
	backgroundcolor=\color{mygray},
	]
queueHandle_button = xQueueCreate(QUEUE_MAX_LENGTH, sizeof(eSystemEvent_button));
if (queueHandle_button == NULL){
    perror("Error creating queue");
    return 1;
}
\end{lstlisting}


Todas las tareas y recursos del sistema operativo se han protegido contra errores informando al usuario ante fallas en la instanciación, previas al inicio del \textit{scheduler} u orquestador.

\begin{lstlisting}[caption=Ejemplo de protección en la creación de tarea al inicio del orquestador.,
	language=C, 
	backgroundcolor=\color{mygray}
	]
if( xTaskCreate( vTaskReadSerial, "Serial Comm reading task", 
    configMINIMAL_STACK_SIZE*4, NULL, tskIDLE_PRIORITY+2, &xTaskReadSerialHandler) 
    == pdFAIL ) {
    perror("Error creating task");
}
\end{lstlisting}

Se utilizan también punteros de tipo \textit{xTaskHandle} para referenciar las tareas del sistema operativo. Este tipo de referencias son de especial utilidad a la hora de eliminar tareas de forma dinámica en tiempo de ejecución, y también se pueden utilizar para la comunicación entre tareas.\\

El uso de \textit{timers} por software fue el preferido en aquellos casos que su uso facilita el mantenimiento, legibilidad y comprensión de la implementación. La plantilla desarrollada para la creación de \textit{timers} se presenta en el siguiente fragmento de código.

\begin{lstlisting}[caption=Ejemplo de creación de temporizadores por software.,
	language=C, 
	backgroundcolor=\color{mygray},
	]
void timerCallback_displayled(TimerHandle_t xTimerDisplayHandle){

   if (pdTRUE == xSemaphoreTake( xMutexUART, portMAX_DELAY)){
      printf("Timer display led is running.\r\n");
      xSemaphoreGive(xMutexUART);
   }

   eSystemEvent_displayled  displayled_timer_event = evDisplayled_timeout;
   
   if(xQueueSend(queueHandle_displayled, &displayled_timer_event, 0U)!=pdPASS){
         perror("Error sending data to the queueHandle_displayled\r\n");
   }
}

\end{lstlisting}

Se procuró tener especial cuidado de superposición o competición del reloj del sistema operativo entre tareas. Distintos componentes pueden tener requisitos temporales que compiten por la prioridad del orquestador. Para ilustrar este aspecto, se detallará la implementación del display led como un ejemplo que puede competir con el arribo de mensajes por la interfaz UART.\\

Las rutinas de interrupción por hardware, conocidas como \textit{ISR}, se utilizan cuando se requiere respuesta inmediata. Por ejemplo, al accionar manualmente un interruptor o bien al recibir mensajes o señales eléctricas desde el bus de datos del tren. La implementación elegida para el uso de interrupciones se representa en el diagrama de secuencia de la figura \ref{fig:ISR}. \\

\begin{figure}[ht]
	\centering
	\includegraphics[width=1\textwidth]{./Figures/ISRcallback.png}
	\caption{Diagrama de secuencia que representa la interacción entre componentes del sistema operativo cuando se dispara una interrupción o ISR.}
	\label{fig:ISR}
\end{figure}

El recurso de hardware que genera interrupciones (ISR) llama a una función \textit{callback} (\textit{ISR callback}), que sólo se encarga de entregar un mensaje de evento al sistema. Esta función no tiene lógica implementada, sólo avisa al sistema que hay una interrupción que atender. El evento se comunica a través de una cola, recurso elegido como interfaz de mensajes de los objetos activos. La cola es atendida por una tarea de sistema (\textit{vTask}) que resuelve el código de ejecución. Esta tarea normalmente implementa un objeto activo que actualiza una máquina de estado.\\

Aunque la complejidad del mecanismo de atención de interrupciones puede resultar llamativa, este patrón basado en objeto activo permite desacoplar la invocación y la ejecución de una máquina de estados. El objeto activo es una técnica de concurrencia muy utilizada debido a su capacidad para separar en hilos independientes (\textit{threads}) la invocación de funciones (eventos) de las funciones de ejecución. Con este mecanismo aparece la posibilidad de atender múltiples eventos en simultáneo, sin esperar a que se procese cada evento secuencialmente. Esto permite lograr paralelismo, garantizar la atomicidad en la función que atiende la interrupción, mantener una latencia muy baja y asegurar la consistencia con en el uso de objetos activos bajo los lineamientos de la arquitectura orientada a eventos.\\


Por último, se menciona el uso de memoria dinámica. El sistema cuenta con una interfaz de comunicación UART que, en principio, recibirá mensajes de longitud variable y desconocida. Por lo tanto, se optó por utilizar \textit{buffers} dinámicos para el almacenamiento y procesamiento de datos. Para el manejo de memoria dinámica se hace uso de funciones como \textit{memset()}, \textit{pvPortMalloc()}, \textit{memcpy()} y \textit{vPortFree()}, a través de dos instancias de ejecución: una para la recepción y el almacenamiento de datos, y otra para la ejecución y liberación de memoria. En el siguiente fragmento de código se detalla el uso de estas funciones en el contexto de dos tareas: \textit{reader} y \textit{writer}.


\begin{lstlisting}[caption=Ejemplo de memoria dinámica con dos tareas reader y writer.,
	language=C, 
	backgroundcolor=\color{mygray},
	]
void vTaskReader(void *parameters) {
   // Clear whole buffer
   memset(buf, 0, buf_len);
   // Loop forever
   while (true) {
   // Read from UART
      uint8_t c_data=0;
      if (uartReadByte( UART_USB, &c_data ) == true ){
      // Store to buffer if not over buffer limit  
         if (idx < buf_len - 1){
            buf[idx] = c_data;
            idx++;
         }
         // Create a message ending string with null
         if (c_data == '\n') {
            buf[idx - 1] = '\0';
            // Allocate memory and copy message. 
            if (msg_flag == 0) {
               msg_ptr = (char *)pvPortMalloc(idx * sizeof(char));
               if (msg_ptr==NULL){
                  if (pdTRUE == xSemaphoreTake( xMutexUART, portMAX_DELAY)){
                     printf("Buffer out of memory\r\n");
                     xSemaphoreGive(xMutexUART);
                  }
               }
               // Copy message
               memcpy(msg_ptr, buf, idx);
               // Notify other task that message is ready
               msg_flag = 1;
            }
            // Reset buffer and index
            memset(buf, 0, buf_len);
            idx = 0;
         }
      }
   }
}     

void vTaskWriter(void *parameters) {
   if (uart_msg_flag == true) {
      // Print the message in the buffer 
      if (pdTRUE == xSemaphoreTake( xMutexUART, portMAX_DELAY)){
        printf("%s\r\n", uart_msg_ptr);
        xSemaphoreGive(xMutexUART);
      }
      // Free the memory block  
      vPortFree(uart_msg_ptr);  
      uart_msg_ptr  = NULL;
   }
}

\end{lstlisting}

Como se puede observar, cuando se recibe un carácter por la interfaz UART, se copia este carácter en el \textit{buffer}, siempre y cuando haya espacio disponible. Una vez que el \textit{buffer} se llena, se solicita memoria adicional y se copia el mensaje utilizando un puntero al bloque de memoria asignado. Luego, la tarea de ejecución lee el mensaje de ese puntero a memoria, o bien lo procesa para invocar otra función.\\


\subsection{Patrones de software}
En este desarrollo se ha hecho uso extensivo de los patrones máquina de estado, objeto activo, \textit{pipeline}, observar y reaccionar, y superciclo. Se presenta en detalle el formato de plantillas desarrollado en C para freeRTOS para los patrones de máquinas de estado y de objeto activo. \\

\subsubsection{Máquinas de estado}
La arquitectura orientada a eventos propuesta se desarrolla a través de la interacción de máquinas de estado mediante el uso de objetos activos. En este trabajo, la implementación de las máquinas de estado se lleva a cabo de manera sistemática en el lenguaje C, siguiendo un procedimiento paso a paso que se describe a continuación. Como ejemplo, se utiliza una máquina de dos estados A y B, que cambia de un estado a otro en respuesta a un evento temporal (\textit{evTimeout}):

\begin{enumerate}
\item Representación de los estados y eventos con diagrama de estados.
\begin{figure}[ht]
	\centering
	\includegraphics[width=0.66\textwidth]{./Figures/statemachineAB.png}
	\caption{Diagrama de la máquina de estados AB ejemplo.}
	\label{fig:fsmAB}
\end{figure}

\item Definición de los estados, usando tipo enumerativo definido con el prefijo \textit{eSystemState}:

\begin{lstlisting}[caption=Definición de estados.,
	language=C, 
	backgroundcolor=\color{mygray},
	]
typedef enum{
    
    STATE_INIT,
	STATE_A,
	STATE_B

} eSystemState;
\end{lstlisting}

\item Definición de los eventos. Los eventos representan transiciones entre estados con un tipo enumerativo definido con el prefijo \textit{eSystemEvent}:

\begin{lstlisting}[caption=Definición de eventos.,
	language=C, 
	backgroundcolor=\color{mygray},
	]
typedef enum{

	evInit,
	evTimeout

} eSystemEvent;
\end{lstlisting}

\item Definición de un tipo puntero a función usando el prefijo \textit{*pfEventHandler()} para designar los handlers específicos:

\begin{lstlisting}[caption=Definición de puntero a handlers.,
	language=C, 
	backgroundcolor=\color{mygray},
	]
typedef eSystemState (*pfEventHandler)(void);
\end{lstlisting}

\item Definición de una estructura para la máquina de estados definiendo un tipo \textit{sStateMachine}. Esta estructura debe incluir una variable estado (\textit{eSystemState}), una variable evento (\textit{eSystemEvent}) y un puntero a función (\textit{pfEventHandler}). El puntero a función será una instancia de \textit{handler} específico que maneje las transiciones entre estados. 

\begin{lstlisting}[caption=Estructura para máquina de estados.,
	language=C, 
	backgroundcolor=\color{mygray},
	]
typedef struct{

	eSystemState  	fsmState;
	eSystemEvent  	fsmEvent;
	pfEventHandler	fsmHandler;

} sStateMachine;
\end{lstlisting}

\item Definición de los \textit{handlers} a implementar para el manejo de ejecución y transiciones entre estados.

\begin{lstlisting}[caption=Definición de handlers.,
	language=C, 
	backgroundcolor=\color{mygray},
	]
eSystemState 	InitHandler(void);
eSystemState 	AHandler(void);
eSystemState 	BHandler(void);
\end{lstlisting}

\item Instanciación de la máquina de estados como un arreglo de estructuras usando el prefijo \textit{sStateMachine\_}. Para el ejemplo de la máquina AB, la instancia de arreglo de estructuras sería la siguiente:

\begin{lstlisting}[caption=Ejemplo de instanciación de una máquina de estados.,
	language=C, 
	backgroundcolor=\color{mygray},
	]
sStateMachine_AB fsmMachineAB [] = 
{
	{STATE_INIT_AB, evInit_AB, InitHandler_AB},
	{STATE_A, evTimeout, AHandler},
	{STATE_B, evTimeout, BHandler}
};
\end{lstlisting}

En esta definición habrá un \textit{handler} en el momento de inicialización (\textit{initHandler}) y  un \textit{handler} específico para cada estado.

\item Escribir el código ejecutable de los \textit{handlers}. Una implementación a modo de ejemplo se presenta en el siguiente fragmento de código:

\begin{lstlisting}[caption=Ejemplo de implementación de handlers.,
	language=C, 
	backgroundcolor=\color{mygray},
	]
eSystemState 	InitHandler(void){ 
	printf("State Machine Init...\n");
	return STATE_A; 
}

eSystemState 	AHandler(void){ 
	printf("State Machine State A\n");
	return STATE_B; 
}

eSystemState 	BHandler(void){ 
	printf("State Machine State B\n");
	return STATE_A; 
}
\end{lstlisting}
\end{enumerate}


Se debe notar que para este ejemplo cada transición de estados se ejecutará por el evento \textit{evTimeout}. La máquina inicia y se define en el estado STATE\_A y luego alterna entre \textit{STATE\_A} y \textit{STATE\_B} por cada evento de timeout.

De esta manera, queda desacoplada la implentación de los \textit{handlers} del resto de la estructura de la máquina de estados, logrando portabilidad, escala y modularidad. Los \textit{handlers} serán funciones que se implementan con el sufijo \textit{Handler()} y que por definición tienen un solo argumento de tipo \textit{void}.

Con esta técnica de desacoplamiento, se implementan dos archivos por cada máquina de estado : uno de encabezados (\textit{stateMachine.h}) con las definiciones, y otro con la implementación (\textit{stateMachine.c}) de los handlers.




\subsubsection{Objeto activo}
	
Los objetos activos se implementan en esta aplicación como tareas de freeRTOS, usando dos superciclos anidados de tipo \textit{while(true)}, tal como se muestra en el siguiente fragmento de código: 

\begin{lstlisting}[caption=Plantilla de implementación para objetos activos.,
	language=C, 
	backgroundcolor=\color{mygray},
	]
void vTaskAB(void *xTimerHandle)
{
   (void)xTimerHandle;

   // Task successful creation message
   if (pdTRUE == xSemaphoreTake( xMutexUART, portMAX_DELAY)){
      printf("Task AB is running.\r\n");
      xSemaphoreGive(xMutexUART);
   }

   while(true){
      
      // State machine init
      eSystemEvent_AB newEvent	=	evInit_AB;
      eSystemState_AB nextState	=	STATE_INIT_AB;
      fsmMachineAB[nextState].fsmEvent = newEvent; 
      nextState = (*fsmMachineAB[nextState].fsmHandler)();

      // Active object
      while(true){
        if( pdPASS == xQueueReceive(queueHandle_AB, &newEvent, portMAX_DELAY)){
            fsmMachineAB[nextState].fsmEvent = newEvent; 
            nextState = (*fsmMachineAB[nextState].fsmHandler)();
         }
      }
   }
}
\end{lstlisting}

Es importante notar que el primer ciclo \textit{while()} es el que corresponde al funcionamiento normal de una tarea o proceso de RTOS. En este caso, es el encargado de inicializar la máquina de estados asociada al objeto activo. El ciclo \textit{while()} de la línea 20 se encarga de bloquear la tarea hasta que se reciba un evento por la interfaz (cola de eventos). La interfaz del objeto activo es una cola FIFO (\textit{First In First Out}) con la función de sistema \textit{xQueueReceive() }(línea 21). Si se recibe un evento, entonces se actualiza el estado de la máquina instanciando el handler que corresponda, como se observa en la línea 25. El diagrama de la figura \ref{fig:AOfsmAB} representa el objeto activo detallado. 

\begin{figure}[ht]
	\centering
	\includegraphics[width=0.85\textwidth]{./Figures/AOstatemachineAB.png}
	\caption{Diagrama del objeto activo de la máquina de estados AB ejemplo.}
	\label{fig:AOfsmAB}
\end{figure}

Para este ejemplo, la tarea de sistema recibe una referencia a un timer (\textit{xTimerHandle}) que genera eventos de \textit{timeout} y los encola en la interfaz del objeto activo (\textit{Queue}). Con la misma interfaz, los eventos también podrían ser generados por otros objetos activos, o por rutinas de interrupción. Los mensajes en cola los recibe  la tarea de sistema que actualiza la máquina de estados (\textit{vTaskAB}).\\

\subsection{Componentes del sistema}

En esta sección, se describen los componentes principales de la solución representada con el diagrama estructural de la figura \ref{fig:diagVistaDisenho}. En la figura \ref{fig:diagramaSecuenciaSistema} se presentan las interacciones entre componentes en forma de secuencia para los distintos casos de uso. Los cuatro objetos activos utilizados son:\\
\begin{itemize}
\item \textbf{UART}: objeto activo que sirve de interfaz de comunicación con la red RS485 del sistema PIDS. Se encarga de recibir los mensajes de la red, identificarlos y enviarlos al objeto PIDS.
\item \textbf{PIDS}: objeto activo que contiene la lógica de procesamiento de señales del tren, los mensajes que se visualizan en pantalla y los trayectos disponibles. 
\item \textbf{displayled}: objeto activo responsable de codificar los mensajes que vienen del PIDS para ser visualizados en un display de matriz led.
\item \textbf{Button}: objeto activo responsable de recibir accionamientos manuales del conductor del tren.
\end{itemize}

\begin{figure}[ht]
	\centering
	\includegraphics[width=1\textwidth]{../Figures/secuenciasSistema.png}
	\caption{(a) Caso de uso de visualización de estación; (b) Caso de uso de receptor de tramas; (c) Caso de uso de elección de destino por accionamiento del conductor; (d) Caso de uso de visualización de información de asistencia.}
	\label{fig:diagramaSecuenciaSistema}
\end{figure}


%La organización del orquestador del sistema operativo se representa con el diagrama temporal de la figura \ref{fig:diagramasTemporales}. La base de este diagrama es el orden de prioridades de las tareas de objetos activos, los timers y los handlers de interrupciones que manejan los periféricos de hardware.\\

\subsubsection{Periférico UART}

 Este tipo de interfaz suele ser común en numerosas aplicaciones y abundan ejemplos utilizados para leer y escribir desde y hacia un periférico UART. El caso trivial es una aplicación 'eco': todo lo que se recibe por la UART se vuelve a escribir y enviar por la UART. Sin embargo, la aplicación específica determina la lógica de procesamiento de mensajes. Por ejemplo, si se recibe un caracter determinado, entonces se activa tal objeto; si se genera un evento en tal otro objeto, entonces se envía un mensaje de aviso. \\

En el sistema desarrollado, la UART es la interfaz de comunicación con el resto de la red PIDS. Esta debe procesar eventos que indican aceleración y desaceleración del tren. Como se verá en la sección de ensayos, se ha observado que las tramas normalmente tienen un encabezado (\textit{header}), una carga de datos (\textit{payload}), y un final de trama (\textit{trailer}). En los ensayos también se ha observado que los mensajes transmitidos en la red PIDS de los trenes de SOFSE pueden tener largo variable. El diseño de este componente se representa con el diagrama de la figura \ref{fig:diagfsmUART}.\\


\begin{figure}[ht]
	\centering
	\includegraphics[width=1\textwidth]{./Figures/fsmUART3.png}
	\caption{Diagrama del objeto activo UART implementado.}
	\label{fig:diagfsmUART}
\end{figure}

El periférico UART es un componente de bajo nivel del microcontrolador que admite el control por interrupciones (ISR). Cada byte entrante genera una interrupción y de inmediato el \textit{handler} \textit{IRQ\_UART} envía un mensaje de actualización al objeto activo. Los eventos admitidos son:
\begin{itemize}
\item \textit{evUart\_Received\_byte} para la recepción de un byte de datos; 
\item \textit{evUart\_Timeout}, generado por un timer específico de control.\\
\end{itemize} 

La máquina de estados admite cuatro estados distintos:
\begin{itemize}
\item \textit{IDLE}: estado inicial y de reposo.
\item \textit{LISTENING}: estado generado al recibir el Header o inicio de trama.
\item \textit{VALID}: estado generado al completar un mensaje con el Trailer o final de trama.
\item \textit{ERROR}: estado alcanzado una vez generado un timeout en el estado listening.
\end{itemize}

El estado \textit{LISTENING} es el \textit{core} de la máquina de estados. El \textit{handler} de ejecución permite almacenar en memoria dinámica el contenido de un mensaje de longitud variable una vez que se recibe un byte de inicio de trama.  El mensaje se completa cuando se recibe un byte de final de trama, que genera mediante \textit{timeout} el evento de transición al estado \textit{Valid}. La lógica detrás de este diseño es que, una vez que se valida una trama de datos, se genera un evento de actualización hacia el componente externo PIDS. De esta manera, se desarrolló un componente UART con flexibilidad, que permite cambiar los bytes de \textit{header} y \textit{trailer} y admite un \textit{buffer} de longitud variable usando memoria dinámica. \\

\subsubsection{Display led}

En esta sección se describe la solución implementada para controlar carteles de matriz led. La estructura de control, el conjunto de chips compatibles, el uso y posible reutilización en otros sistemas se presenta con diagramas y detalles de código fuente. \\

Los carteles led utilizados en este trabajo se basan en una tecnología de microcontrolador (MCU) y sistema digital. El MCU genera los mensajes a visualizar en formato de datos binarios, y el sistema digital recibe los datos binarios y señales de control para prender los leds del cartel de forma ordenada. Los carteles se componen de módulos matriciales de 8x8 leds, es decir, 8 filas de 8 leds por fila cada módulo. Los arreglos de módulos forman paneles, y los arreglos de paneles forman carteles. Los módulos  pueden ser controlados a través de pulsos de tensión sincronizados entre filas y columnas. Sin embargo, si se quisiera controlar varios módulos de matriz led en simultáneo, la cantidad de señales a priori podría ser proporcional al número total de leds. El sistema digital expone una solución para que los paneles, como arreglos de módulos, puedan ser direccionados por un grupo de tres o cuatro señales de control usando técnicas de multiplexación y registros de desplazamiento, o \textit{Shift Registers}, como se describe a continuación. \\


\begin{figure}[ht]
	\centering
	\includegraphics[width=1\textwidth]{./Figures/diagDriverled.png}
	\caption{Diagrama de bloques del controlador de los carteles de matriz led utilizados en esta implementación.}
	\label{fig:diagDriverled}
\end{figure}

En la figura \ref{fig:diagDriverled} se presenta un diagrama de bloques del sistema digital de control para carteles de matriz led basados en el \textit{chipset} 74HC245, 74HC595 y 74HC138. Se pueden distinguir los siguientes bloques:

\begin{itemize}
\item \textit{MCU}: se encarga de generar y transmitir las señales del sistema a través del conector de entrada del cartel de matriz led.
\item \textit{input connector}: conector de pines en el cartel para señales de entrada (\textit{Data, Clock, Latch}) provenientes del MCU.
\item \textit{output connector}: conector de pines en el cartel para señales de salida, para interconexión en serie con otro cartel.
\item \textit{Buffer}: adaptador de nivel de la señal de tensión y derivador a izquierda o derecha de la placa física.
\item \textit{Shift Registers}: registros de desplazamiento para enviar datos binarios a las columnas de los paneles.
\item \textit{Deco}: doble decodificador 3x8 para habilitar secuencialmente las filas de los paneles.
\item \textit{MOSFET Array}: circuito de corriente para energizar las filas de los paneles.
\item \textit{Led dot matrix array}: el cartel de matriz led propiamente dicho.
\end{itemize}

El funcionamiento del circuito es el siguiente. Los mensajes y señales de control (\textit{data}, \textit{clock}, \textit{latch}, \textit{deco}) se generan en el MCU y se transmiten al cartel a través de un conector de entrada (\textit{input connector}). Las señales de control se direccionan a izquierda y derecha a través de \textit{buffers} de la serie 74HC245D, que además entregan un nivel lógico de salida consistente, por ejemplo de 5 Volt. Las señales que van para el lado izquierdo contienen los datos binarios a visualizar en el cartel. A través de los registros de desplazamiento, los datos binarios (\textit{data}) se cargan bit a bit sincronizados con cada ciclo de reloj (\textit{clock}), hasta cargar los leds de una fila completa del cartel. Luego se envía un pulso de \textit{latch} para descargar la fila y habilitar los registros de desplazamiento para los datos de la siguiente fila. Por el lado derecho del \textit{buffer},  un par de decodificadores 74HC138 permiten encender hasta 16 salidas secuencialmente. En cada ejecución de la señal \textit{deco} se energiza una fila usando transistores (\textit{MOSFET Array}), que entregan la corriente necesaria para que todos los leds de cada fila puedan brillar con intensidad. La secuencia coordinada de enviar los datos a una fila, energizarla y luego hacer lo mismo con la fila siguiente, se repite hasta completar todas las filas del cartel en ciclos de 20 ms. De esta manera, se transmite un mensaje completo al cartel aproximadamente 50 veces por segundo, formando una imagen continua vista por el ojo humano.\\

 Para armar carteles más grandes, ses común conectar el conector de salida de un cartel al conector de entrada de otro idéntico, logrando conexiones en serie entre paneles. La lógica de codificación de mensajes es la misma, logrando que un solo panel se conecte al MCU y el resto siga una conexión en cascada, usando el mismo grupo de señales generadas.\\
 
 
Los datos que se generan en el MCU para visualizar mensajes responden a un procesamiento ordenado de información. En la implementación desarrollada, se debe transformar el mensaje a visualizar como texto plano, y luego codificarlo en una matriz de unos y ceros que tenga las dimensiones del cartel de matriz led. A modo de ejemplo, si se codificara un caracter por módulo, se podría enviar mensajes de hasta 16 caracteres con un arreglo de 16 módulos 8x8. Cada carácter requiere de un mapa de bits que permita asociarlo con una matriz de datos binarios de 8x8. En la figura \ref{fig:dataPipeline} se muestra un ejemplo con la letra 'H' codificada por la función f1, y se presenta esquemáticamente la implementación desarrollada. Se han diseñado dos funciones f1 y f2 usando el patrón \textit{pipeline} (f1 y f2) para preprocesar mensajes de texto en datos binarios compatibles con el formato de carteles de matriz led. La función f1 se llama \textit{"string\_read\_to\_8x8\_bytes\_out()"}, se encarga de recibir un mensaje en texto plano, mapearlo a un diccionario de caracteres y matrices, y entregar un arreglo de números que corresponden al mapa binario de 8x8 de cada caracter. Así, por cada caracter que recibe f1 se generan 8 números binarios. Luego, la función f2 llamada \textit{"reshape\_to\_display()"} recibe como argumentos las dimensiones del cartel y reordena el arreglo de números binarios acorde al formato del cartel. Esta última función es esencialmente una operación de transposición de matriz y considera tres casos posibles: si el mensaje a visualizar es más corto que el espacio del cartel, si es más largo, o si tiene la misma longitud. En cada caso, la función ajusta la matriz transpuesta, llenando con ceros cuando es necesario.\\

\begin{figure}[htbp]
	\centering
	\includegraphics[width=1\textwidth]{./Figures/dataPipeline.png}
	\caption{Procesamiento de datos para codificar mensajes.}
	\label{fig:dataPipeline}
\end{figure}

La pieza de software desarrollada para el control del display led es un objeto activo consistente con el resto del sistema. El diagrama de la máquina de estados asociada se presenta en la figura \ref{fig:fsmDisplayled}. Se pueden distinguir tres estados, cada uno con un handler asociado. El estado \textit{IDLE} es el estado inicial del sistema y es también el estado de reposo cuando no hay información para visualizar en el cartel. Ante un evento de mensaje recibido (\textit{evDisplayLed\_msg\_received}), hay una transición al estado \textit{PROCESSING} y se ejecuta el \textit{pipeline} descrito para transformar texto plano en matrices de unos y ceros. Finalizado el procesamiento, se entrega al sistema digital para energizar el cartel y visualizar la información dentro del estado \textit{ENCODING}.\\

\begin{figure}[htbp]
	\centering
	\includegraphics[width=0.66\textwidth]{./Figures/FSMdisplayled.png}
	\caption{Diagrama de estados para la máquina de estados del display led.}
	\label{fig:fsmDisplayled}
\end{figure}

En los siguientes fragmentos de código se detallan los \textit{handlers} desarrollados para la máquina de estados presentada. El handler \textit{displayled\_procHandler()} implementa el pipeline de las funciones f1 y f2. Se puede observar que luego del mapeo de caracteres a arreglos de bytes en la línea 9, el reordenamiento de los bytes implica pasar de una matriz de dimensiones nxm, con 'n' la cantidad de filas por caracter y 'm' el largo del mensaje, a una matriz pxq, con 'p' la cantidad de filas del display led y 'q' la cantidad de columnas. El ciclo de la línea 19 inicializa en cero la nueva matriz y en la línea 22 se completa el pipeline con la función f2.\\

\begin{lstlisting}[caption=Handler para procesamiento de datos de cartel de matriz led.,
	language=C, 
	backgroundcolor=\color{mygray},
	caption=	{Código fuente del handler de procesamiento del display led.},
	captionpos=b]
eSystemState_displayled     displayled_procHandler(void){

    char *str1=messages[displayled_msg_idx];

    uint8_t str1_len=strlen(str1);
    uint8_t buffer_size=str1_len*CHAR_LENGTH;
    uint8_t buffer[buffer_size];

    string_read_to_8x8_bytes_out(str1,str1_len,buffer);

    int n=CHAR_LENGTH; 
    int m=str1_len;
    int p=DISPLAYLED_ROWS;
    int q=DISPLAYLED_COLS;

    int displayled_size = p*q;

    uint8_t B[displayled_size];
    for(int i=0; i<displayled_size;i++)
    B[i]=0;

    reshape_to_display(buffer, displayled_buffer, buffer_size, displayled_size);

    displayled_msg_idx++;
    displayled_msg_idx%=MESSAGES_TOTAL_NUMBER;
    displayled_msg_flag=0;

    return STATE_DISPLAYLED_ENCODING;
}

\end{lstlisting}

El \textit{handler} \textit{diplayLed\_dataHandler()} implementa la lógica del sistema digital asociado al cartel, detallado al principio. Se pueden observar dos ciclos for anidados en las líneas 12 y 16, uno para recorrer caracter a caracter el mensaje, y otro para escanear bit a bit cada caracter. La línea 19 implementa una máscara de bit a bit para transmitir el dato al registro de desplazamiento, y las líneas 33-34 generan el pulso de \textit{latch}. La secuencia codificada de las líneas 36 en adelante implementan la habilitación fila a fila usando decodificadores 3 a 8. \\

\begin{lstlisting}[caption=Handler para codificación de datos en cartel de matriz led.,
	language=C, 
	backgroundcolor=\color{mygray},
	caption=	{Código fuente del handler de encoding del display led.},
	captionpos=b]
eSystemState_displayled     displayled_dataHandler(void){

    uint8_t data_8b;
    bool_t  value;

    displayled_timer_cnt--;
    if(!displayled_timer_cnt){
        displayled_msg_flag=0;
        return STATE_DISPLAYLED_IDLE;
    };    

    for(int i=0; i<displayled_size; i++){
        data_8b = displayled_buffer[i];
        for(int j=0; j<8; j++){
            // displayled_data 
            value = (((data_8b << j ) & 0x80 ) == 0) ? 1 : 0;
            printf("%d",value);
            gpioWrite(displayled_panel_1, value);
            gpioWrite(displayled_panel_2, value);
            // displayled_clock 
            gpioWrite(displayled_clk, ON);
            gpioWrite(displayled_clk, OFF);
        }
        
        if(i%DISPLAYLED_COLS==0){
            // displayled_latch 
            gpioWrite(displayled_latch, ON);
            gpioWrite(displayled_latch, OFF);
            // displayled_row_scanning
            displayled_deco_cnt++;
            displayled_deco_cnt%=DISPLAYLED_ROWS;
            if((displayled_deco_cnt%1)==0){ 
            		gpioToggle(displayled_deco_A0); }
            if((displayled_deco_cnt%2)==0){ 
            		gpioToggle(displayled_deco_A1); }
            if((displayled_deco_cnt%4)==0){ 
            		gpioToggle(displayled_deco_A2); }
            if((displayled_deco_cnt%8)==0){ 
            		gpioToggle(displayled_deco_A3); }
        }
    }
    
    return STATE_DISPLAYLED_ENCODING;
}

\end{lstlisting}

Con esta implementación de controlador de matriz led, ha sido posible realizar distintas pruebas de funcionamiento que se detallan en el capítulo siguiente.\\

% Chapter Template

\chapter{Ensayos y resultados} % Main chapter title
En este capítulo se detallan los ensayos realizados en las formaciones ferroviarias y en los talleres de Trenes Argentinos. El orden cronológico de los ensayos es distinto al del desarrollo del firmware. En este documento se ha presentado previamente el diseño de la solución para facilitar la comprensión del trabajo realizado. El desarrollo de la solución fue posterior a una serie de mediciones realizadas en los talleres que permitieron identificar parámetros clave del sistema. \\

En las secciones que siguen se explican las mediciones realizadas en las visitas a los talleres de Victoria y Castelar de Trenes Argentinos Operaciones. Luego se presenta un análisis de datos de las tramas relevadas y también las pruebas de integración propuestas para validar el desarrollo. \\


\label{Chapter4} % Change X to a consecutive number; for referencing this chapter elsewhere, use \ref{ChapterX}

%----------------------------------------------------------------------------------------
%	SECTION 1
%----------------------------------------------------------------------------------------
\section{Ensayos en trenes}

Uno de los objetivos generales del proyecto en el que se enmarcó este trabajo, era relevar las soluciones existentes de la red TCN y PIDS. A lo largo del desarrollo, se realizaron reuniones de trabajo con el personal de Trenes Argentinos Operaciones, y visitas a los talleres de la Gerencia de material rodante eléctrico para relevar información técnica. En orden cronológico, incluyendo trabajo realizado en etapa de confinamiento por COVID-19, se resaltan las siguientes interacciones con SOFSE:\\


\begin{enumerate}

\item 2020-05-20: se ensayaron mediciones sobre la maqueta de los talleres de Castelar, coordinadas de forma remota, para obtener mediciones de datos de la red RS485 del sistema PIDS.

\item 2020-06-30: se ensayaron mediciones sobre formaciones ferroviarias operativas en los talleres de Victoria, relevando los puntos de interconexión de la red TCN con el TLCD (pantalla táctil del conductor). Se realizaron también mediciones en el punto de interconexión del bus MVB con el módulo RCMe. 

\item 2020-09-23: se ensayaron nuevas mediciones, coordinadas de forma remota, en la maqueta de Castelar con un nuevo módulo analizador de datos.

\item 2021-04-09: se relevaron los puntos de interconexión entre RCMe y el PIDS, y entre los módulos IDU y SCU en formaciones operativas en los talleres de Victoria. También se ensayaron mediciones sobre el bus de datos RS485 que conecta el SCU con la DACU.

\item 2021-06-04: se ensambló una maqueta local usando un cartel led compatible con la serie de los carteles de trenes relevados y se presentó como informe de avance.

\item 2022-08-11, CASE 2022.

\end{enumerate}


En la figura \ref{fig:maquetaCastelar} se muestra la maqueta instalada en los talleres de Castelar. Se puede observar un rack con los módulos del sistema PIDS, incluyendo la pantalla LCD táctil que maneja el conductor. También se observan los carteles de matriz led frontal (cartel grande), de salón (cartel chico) y el mapa led con el recorrido de las estaciones (cartel del medio).\\
 
\begin{figure}[H]
	\centering
	\includegraphics[width=0.66\textwidth]{./Figures/maqueta.png}
	\caption{Maqueta en talleres de Castelar.}
	\label{fig:maquetaCastelar}
\end{figure}


Durante las visitas a los talleres de Victoria, se pudo acceder a los planos eléctricos del sistema de comunicaciones del tren. Esta información resultó muy relevante ya que permitió comprender la lógica de interconexión de los módulos del tren y preparar un sistema de medición. Como consecuencia de las visitas, se fueron desarrollando piezas sueltas de hardware para realizar mediciones in-situ, a partir del relevamiento de los conectores y conexiones entre módulos.\\


Las redes del sistema PIDS y TCN siguen el estándar RS-485. Este tipo de redes es muy utilizada para transmisión y recepción de datos, ya que tiene interfaces eléctricas muy robustas que usan señales diferenciales, y que normalmente se implementan en cables de par trenzado, permitiendo cableados largos con buena inmunidad al ruido eléctrico. \\

En las figuras \ref{fig:conexionOriginal} y \ref{fig:conexionIntervenida} se presenta un diagrama esquemático del punto de medición y una fotografía de la medición realizada en el sitio, esto es, en una formación operativa en los talleres de Victoria. La conexión original muestra un grupo de tres cables nomenclados como 4330a, 4330b y 4330s, que corresponden con las líneas del bus RS485, RS485a, RS485b y RS485c respectivamente, que conecta el SCU con la placa (IDU) de la placa de control del cartel de matriz led.  La intervención en el punto de medición se realizó a través de una placa fabricada ad-hoc. Esta placa cuenta con conectores Harting de entrada y salida, facilitados por el personal de SOFSE, conversores RS485-USB y un analizador lógico programable, utilizado para decodificar en tiempo real los datos medidos.\\



\begin{figure}[H]
	\centering
      \includegraphics[width=0.25\textwidth]{./Figures/conexionOriginal.png}
      \includegraphics[width=0.5\textwidth]{./Figures/rackPIDS2.jpg}
	\caption{Diagrama esquemático y fotografía del punto de medición en la conexión SCU-IDU.}
	\label{fig:conexionOriginal}
\end{figure}

\begin{figure}[H]
	\centering
      \includegraphics[width=0.4\textwidth]{./Figures/conexionIntervenida.png}
      \includegraphics[width=0.5\textwidth]{./Figures/setupExperimentalMediciones.jpg}
	\caption{Diagrama esquemático y fotografía de la intervención para realizar mediciones en la conexión SCU-IDU.}
	\label{fig:conexionIntervenida}
\end{figure}




\begin{figure}[H]
	\centering
	\includegraphics[width=0.66\textwidth]{./Figures/sniffer.jpg}
	\caption{Pieza de hardware desarrollada ad-hoc para realizar mediciones.}
	\label{fig:sniffer}
\end{figure}




\begin{figure}[H]
	\centering
	\includegraphics[width=1\textwidth]{./Figures/medicionesVictoria2.png}
	\caption{Fotos de la jornada de mediciones en los talleres de Victoria.}
	\label{fig:medicionesVictoria2}
\end{figure}

La figura \ref{fig:placa} muestra una fotografía del hardware de control de los carteles de matriz led de Trenes Argentinos. Esta placa fue revisada en detalle y se han relevado los siguientes bloques de control:
\begin{itemize}
\item Conector de entrada del bus RS485.
\item Módulos de conversión de tensión.
\item Circuito de optoacopladores para las señales de datos.
\item Microcontrolados y circuito lógico.
\item Conector de salida para el cartel de matriz led.
\item Conector de programación.
\end{itemize}

El circuito eléctrico completo de esta placa ha sido relevado y se puede encontrar en el apéndice. Su función principal es la de decodificar las señales del tren y transmitir los mensajes al cartel de matriz led. El bloque de conversión de tensión contiene varios módulos, que son fuentes conmutadas para transformar la tensión de línea del tren de 110 Volt de corriente contínua en tensiones compatibles con el circuito de datos, por ejemplo 5 Volt o 3,3 Volt.\\

En la figura \ref{fig:mediciones} se muestra una fotografía del banco de medición operando en vivo en una de las formaciones ferroviaras operativas en los talleres de Victoria. Se puede observar la pieza de hardware desarrollada conectada al SCU de un lado, y a la laptop del otro. En la pantalla de la computadora se observan las líneas de datos capturadas por el analizador lógico programable.\\

\begin{figure}[H]
	\centering
	\includegraphics[width=0.66\textwidth]{./Figures/mediciones.jpg}
	\caption{Foto del banco de prueba midiendo en vivo en los talleres de Victoria.}
	\label{fig:mediciones}
\end{figure}



Según la topología de la red PIDS vista, los buses de datos siguen un esquema de tres cables por bus, nomenclados como RS485-A, RS485-B y RS485-C. En particular, la conexión entre el módulo SCU y los carteles de matriz led es por medio del bloque IDU, con los cables 4330a, 4330b y 4330c, números que corresponden al cableado físico. La conexión del SCU al IDU está incluída en un conector Harting de 48 pines, donde se unen varios buses o cableados RS485 de tres líneas.  \\

Los carteles de salón están embebidos en un gabinete de metal, donde se aloja la placa de control y parte del cableado. Esta placa de control
 
\section{Análisis de mediciones}

Los datos relevados en las sucesivas mediciones realizadas han aportado información.\\
Se ha obtenido información acerca del contenido de las tramas para la comunicación entre los sistemas TCMS-PIDS. La información de las tramas que el se detalla con el diagrama de la figura \ref{fig:tramasHeaderPayload}


\begin{figure}[H]
	\centering
	\includegraphics[width=1\textwidth]{./Figures/tramasHeaderPayload.png}
	\caption{Contenido de las tramas de datos entre TCMS y PIDS.}
	\label{fig:tramasHeaderPayload}
\end{figure}

En la figura \ref{fig:tramasBitsIDU} se detalla el contenido de los bytes en la comunicación con dirección PIDS-TCMS. Estas son tramas de longitud mucho mayor a las que se dan en el sentido inverso TCMS-PIDS. Se puede observar que los bits de cada byte están referidos a dispositivos específicos del sistema. Se ha resaltado en color los bits que corresponden a los módulos IDU, y que se puede observar según la nomenclatura que existen hasta 18 unidades de estos módulos, agrupados de a pares por cada salón. Son estos bits los que aportan información acerca del estado de los carteles de matriz led de salón.\\


\begin{figure}[H]
	\centering
	\includegraphics[width=1\textwidth]{./Figures/tramasBitsIDU.png}
	\caption{Detalle bit a bit de los bytes con contenido para el PIDS.}
	\label{fig:tramasBitsIDU}
\end{figure}



\begin{figure}[H]
	\centering
	\includegraphics[width=0.5\textwidth]{./Figures/logFile.png}
	\caption{Detalle de mediciones registradas en formato hexadecimal.}
	\label{fig:logFile}
\end{figure}


\begin{itemize}
\item 7EDFFFFFDEEFBFFFBFF7E
\item 7EBDFFFF7FFFFFFFFFFFFFFF39F7E
\item 7EBDFFFF7FFFFFFFFFFFFFFF39F7E
\item 7EBDFFFF7FFFFFFFFFFFFFFF39F7E
\item 7EFFFF7FFFFFFFFFFFFFEFFFA317E
\item 7E19ABF5FFFFFFFFFFFFFFFFFF7F72A85AF7E
\item 7E7CB12914AEDFFFFFFFEFFEFFBFFFFFF7F7FFFFFFFFFDFFFFB77E
\end{itemize}



\section{Hardware existente}

\begin{figure}[H]
	\centering
	\includegraphics[width=0.75\textwidth]{./Figures/displayController.jpg}
	\caption{Fotografía del detalle de conexión de la placa de control de los carteles led de salón.}
	\label{fig:displayController}
\end{figure}


\begin{figure}[H]
	\centering
	\includegraphics[width=0.5\textwidth, angle=90]{./Figures/placaIDU.jpg}
	\caption{Placa de control (IDU) de los carteles de matriz led.}
	\label{fig:placa}
\end{figure}

\begin{figure}[H]
	\centering
	\includegraphics[width=1\textwidth]{./Figures/cartel2x6.jpeg}
	\caption{Placa de los carteles de matriz led.}
	\label{fig:placaDisplay}
\end{figure}


\begin{figure}[H]
	\centering
	\includegraphics[width=0.75\textwidth, angle=270]{./Figures/cartel4x8.jpg}\\
	\includegraphics[width=1\textwidth]{./Figures/cartelledON.jpg}\\
	\caption{Fotografías de placas de control de los carteles de matriz led: (a) placa de 2x6 módulos; (b) placa de 4x8 módulos; (c) vista posterior de la placa de 4x8.}
	\label{fig:picsDriverled}
\end{figure}


En el circuito esquemático de la figura \ref{fig:schDriverled} se presenta el detalle de conexiones eléctricas entre bloques. Se puede observar que a la salida del conector de datos (CONN 2x8) hay dos buffers de la serie 74HC245D que direccionan las señales eléctricas a izquierda y derecha del arreglo de matrices led. A izquierda viajan las señales SER(data), SRCLK (Clock) y XXX (latch) al arreglo de Shift Registers de la serie 74HC595. Por la derecha se maneja la habilitación secuencial de las filas a través de un arreglo de decodificadores 3x8 de la serie 74HC138. Cada salida de los decodificadores se conecta a un driver de corriente en arreglo de transistores MOSFET FDS4953. Estos decodificadores cableados adecuadamente permiten manejar las 32 señales de un cartel de 4x8 módulos led. \\

\section{Pruebas realizadas}

\pagebreak

\section{APÉNDICE}


\begin{figure}[H]
	\centering
	%\includepdf[pages={1}, angle=90]{./Figures/output.driverled.pdf}
	\includegraphics[width=1.66\textwidth, angle=90]{./Figures/output.driverled.pdf}
	\caption{Circuito esquemático de la placa controladora de los carteles de matriz led.}
	\label{fig:schDriverled}
\end{figure}

\begin{figure}[H]
	\centering
	\includegraphics[width=1.66\textwidth, angle=90]{./Figures/output.placaControl.pdf}
	\caption{Circuito esquemático de la placa de control de los carteles LED de salón.}
	\label{fig:schController}
\end{figure} 
% Chapter Template

\chapter{Conclusiones} % 

En este trabajo se abordó un problema tecnológico de la industria ferroviaria argentina, se desarrolló un sistema embebido, y se realizaron ensayos en las instalaciones de Trenes Argentinos (SOFSE) que aportaron información muy importante para el desarrollo del sistema.\\

La problemática de Trenes Argentinos expuesta en este trabajo, está asociada al mantenimiento del sistema de visualización de información al pasajero (PIDS). Este sistema presenta carteles de matriz led en los coches, y forma parte de un sistema de red más grande que se encarga de las comunicaciones del tren (TCN). La red TCN está ampliamente utilizada en el transporte ferroviario mundial, y los estándares internacionales que la especifican tienen un rol clave en la industria. Si bien los protocolos y componentes de la red TCN están definidos en las normas o especificaciones, el sistema PIDS no está incluído en el estándar de la versión de red instalada en los trenes de SOFSE, siendo una solución propietaria de un fabricante con escasa documentación disponible. Esto motivó la realización de una serie de ensayos en las formaciones ferroviarias operativas, y en las maquetas de los talleres de SOFSE, para relevar aspectos técnicos de su implementación aplicando técnicas de ingeniería inversa.\\

A lo largo del trabajo, se expuso la arquitectura del sistema PIDS existente y su relación con la red TCN. En la solución instalada en los trenes, hay una relación física directa entre algunos de los bloques de la arquitectura y el hardware, que fue relevada durante los ensayos. Por el contrario, algunas de las interconexiones de la arquitectura son únicamente a nivel lógico. Se observaron bloques como el SCU o la PCU, que funcionan como concentradores y procesadores de buses de datos, sin tener un módulo único de hardware asociado, sino una serie de equipos funcionando en conjunto en distintas unidades de rack.  Con la información de las interconexiones, se ensayaron mediciones en distintos puntos de prueba: entre el TCMS y el PIDS por ejemplo, entre PCU y SCU, o entre el SCU y el IDU. \\

El análisis de datos de las mediciones, en particular para los puntos de prueba TCMS-PIDS, presentó consistencia para los encabezados y final de trama con la documentación de TCN disponible. Sin embargo, se observó que el comienzo y final de trama era distinto del anterior para el punto de prueba SCU-IDU, donde el IDU corresponde al hardware de control de los carteles de matriz led. Estas observaciones resultaron compatibles con mediciones realizadas por el personal de SOFSE con anterioridad. Si bien esta información fue suficiente para desarrollar requerimientos, se ha observado también que las tramas contienen información adicional de otros bloques del sistema, aparte de los carteles de matriz led. Aunque no quede claro el contenido completo de las tramas, es decir, todos los bits que forman parte de la carga útil de las tramas recibidas o transmitidas por los módulos IDU, las hipótesis propuestas para el desarrollo del sistema embebido son compatibles con el comportamiento observado.\\


El desarrollo del sistema embebido ha permitido proponer una solución que otorga cierto grado de abstracción y versatilidad para la interpretación de tramas de datos de la red PIDS. El sistema está diseñado para recibir tramas de longitud variable, que representan eventos, a través de un periférico UART usando técnicas de memoria dinámica. Los datos recibidos pueden ser validados y procesados para activar visualizaciones en carteles de matriz led, compatibles con los que hay instalados en los trenes. El mecanismo de visualización de mensajes para los carteles de matriz led, está desacoplado de la recepción y el procesamiento de las tramas. Es decir, por un lado el sistema puede recibir y reaccionar a eventos externos, y por otro puede generar visualizaciones de mensajes precargados, como por ejemplo el nombre de una estación.\\

 En el diseño se abordaron cuestiones de concurrencia, implementando una arquitectura orientadad a eventos. Se desarrollaron módulos que interactúan entre sí de forma dinámica y que responden a eventos asincrónicos, como por ejemplo la recepción de datos vía UART. Así, un objeto recibe un mensaje, otro lo procesa, y otro genera la visualización en el cartel de matriz led. Cada módulo fue implementado utilizando patrones de diseño de software como máquinas de estados, a su vez embebidas en objetos activos, utilizando una interfaz de comunicación estándar a través de colas de mensajes.  Al implementar los patrones de diseño en lenguaje C, se buscó construir un conjunto de plantillas que pueda facilitar y satisfacer atributos de calidad de software como modularidad y escalabilidad. Todos los objetos activos implementados funcionan de forma orquestada por un sistema operativo de tiempo real. Las relaciones entre componentes se han presentado con vistas estructurales y de interacciones, siguiendo lineamientos de modelado de software UML. Estas interacciones, cumplen con la solución para los casos de uso detallados en la etapa de requerimientos. Para el desarrollo se ha utilizado la plataforma de hardware EDU-CIAA, la capa de abstracción de hardware firmwareV3, y el sistema operativo de tiempo real freeRTOS. Estas tecnologías son de código abierto y de hardware abierto, es decir que además de estar mantenidas por la comunidad, son de acceso libre y cualquiera las puede usar y mejorar el diseño original. \\


El sistema embebido desarrollado fue probado en una maqueta. Para desarrollar un producto funcional o comercial a partir de este sistema, hace falta realizar una serie de ensayos adicionales de compatibilidad en las instalaciones de trenes. Por un lado, hace falta resolver cuestiones de compatibilidad eléctrica: la red de trenes tiene una tensión de línea de 110 Volts y el circuito del sistema embebido funciona con 5 Volt. Se ha observado en las placas IDU, distintos módulos conversores de tensión para alimentar el circuito que procesa los datos. El desarrollo de un circuito impreso, incluyendo módulos conversores de tensión, ha quedado fuera del alcance de este trabajo. Por otro lado, el sistema se ha diseñado para dar versatilidad a la hora de interpretar las tramas de datos de la red PIDS, sin conocer su contenido con completitud. Para completar el estudio integral de las tramas de la red PIDS, hace falta ajustar y verificar el procesamiento de las tramas, por ejemplo vía simulación.\\


Como prospectiva, aunque se ha observado que la versión instalada de redes TCN y PIDS en los trenes de SOFSE corresponden a redes RS485, al momento de redactar este documento se conoce de la existencia de una nueva norma de TCN basada en redes Ethernet, que incluye al PIDS (nomenclado como PIS) dentro el nuevo estándar. Se propone por lo tanto, explorar este nuevo estándar para comprender aspectos de compatibilidad de sistema en caso de realizar una migración.\\ 

%----------------------------------------------------------------------------------------
%	CONTENIDO DE LA MEMORIA  - APÉNDICES
%----------------------------------------------------------------------------------------

\appendix % indicativo para indicarle a LaTeX los siguientes "capítulos" son apéndices

% Incluir los apéndices de la memoria como archivos separadas desde la carpeta Appendices
% Descomentar las líneas a medida que se escriben los apéndices

%% Appendix A

\chapter{Circuitos esquemáticos} % Main appendix title

\label{AppendixA} % For referencing this appendix elsewhere, use \ref{AppendixA}
\pagebreak
\newpage
\begin{figure}[H]
	\centering
	%\includepdf[pages={1}, angle=90]{./Figures/output.driverled.pdf}
	\includegraphics[width=1.66\textwidth, angle=90]{./Figures/output.driverled.pdf}
	\caption{Circuito esquemático de la placa controladora de los carteles de matriz led.}
	\label{fig:schDriverled}
\end{figure}

\begin{figure}[H]
	\centering
	\includegraphics[width=1.66\textwidth, angle=90]{./Figures/output.placaControl.pdf}
	\caption{Circuito esquemático de la placa de control de los carteles LED de salón.}
	\label{fig:schController}
\end{figure}
%\include{Appendices/AppendixB}
%\include{Appendices/AppendixC}

%----------------------------------------------------------------------------------------
%	BIBLIOGRAPHY
%----------------------------------------------------------------------------------------

\Urlmuskip=0mu plus 1mu\relax
\raggedright
\printbibliography[heading=bibintoc]

%----------------------------------------------------------------------------------------

\end{document}  
